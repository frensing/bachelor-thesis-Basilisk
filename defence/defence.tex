\documentclass{beamer}

% Alternative to \usetheme{claw}
\usepackage{template/beamercolorthemeclaw}
\usepackage{template/beamerfontthemeclaw}
\usepackage{template/beamerinnerthemeclaw}
\usepackage{template/beamerouterthemeclaw}

\usepackage{amsmath, amssymb, amsfonts}% mathematical symbols and the like
\usepackage{amsthm}% definitions, theorems, etc.
\usepackage[colorinlistoftodos]{todonotes}% marking open todos in text/on margins
\usepackage{subfig}% multi-part figures with separate captions per part
\usepackage{url}% render URLs correctly and make them clickable through the hyperref package
\usepackage{longtable}% tables that span multiple pages
\usepackage{booktabs}% tables that actually look good
\usepackage[nolist]{acronym}% consistently use acronyms

\usepackage{pgfgantt}

%Customizing GanttChart
\definecolor{barblue}{RGB}{153,204,254}
\definecolor{groupblue}{RGB}{51,102,254}
\definecolor{linkred}{RGB}{165,0,33}
\renewcommand\sfdefault{phv}
\renewcommand\mddefault{mc}
\renewcommand\bfdefault{bc}
\setganttlinklabel{s-s}{START-TO-START}
\setganttlinklabel{f-s}{FINISH-TO-START}
\setganttlinklabel{f-f}{FINISH-TO-FINISH}
\sffamily
\ganttset
{%
	y unit chart=0.8cm,
	canvas/.append style={fill=none, draw=black!35, line width=.5pt},
%	hgrid={*1{draw=black!35, line width=.1pt}},
	hgrid style/.style={draw=black!35,line width=.1pt},
	vgrid={*1{draw=black!35, line width=.1pt}},
	today rule/.style={draw=black!64,dash pattern=on 3.5pt off 4.5pt,line width=1.5pt},
	today label font=\small\bfseries,
	title/.style={draw=none, fill=none},
	title label font=\bfseries\footnotesize,
	title label node/.append style={below=7pt},
	progress label text={}, % \pgfmathprintnumber[precision=0, verbatim]{#1}\%
	include title in canvas=false,
	bar label font=\mdseries\small\color{black!70},
	bar label node/.append style={left=0cm},
	bar/.append style={draw=none, fill=black!63},
	bar incomplete/.append style={fill=barblue},
	bar progress label font=\mdseries\footnotesize\color{black!70},
	bar height=.5,
	bar top shift=0.1,
	group top shift=0.1,
	group incomplete/.append style={fill=groupblue},
	group left shift=0,
	group right shift=0,
	group height=.5,
	group peaks tip position=0,
	group label node/.append style={left=.6cm},
	group progress label font=\bfseries\small,
	link/.style={-latex, line width=1.5pt, linkred},
	link label font=\scriptsize\bfseries,
	link label node/.append style={below left=-2pt and 0pt}
}





% 1st used for slides footer, 2nd for title slide & PDF metadata
\title[Basilisk]{Basilisk -- Continuous Benchmarking for Triplestores}
% Used for title slide & PDF metadata
\subtitle{}

\date{\today}
% Date with conference name
%\date[\today]{Conference on Fabulous Presentations, 2021}

% One author
% 1st used for slides footer, 2nd for title slide & PDF metadata
\author[Fabian~Rensing]{Fabian Rensing}

% 1st used for slides footer, 2nd for title slide
%\institute[DICE]{DICE Group\\Paderborn University}
% Institute with logo
\institute[]{Supervisor: Prof. Dr. Axel Ngonga\\Paderborn University}

% Two authors from different institutes
% Source: https://www.overleaf.com/learn/latex/Beamer#The_title_page
%\author[\underline{Wilke}, Ngonga]{\underline{Adrian~Wilke}\inst{1} \and Axel~Ngonga\inst{2}}
%\institute[]
%{
%	\inst{1}%
%	DICE Group\\
%	Department of Computer Science\\
%	Paderborn University, Germany
%	\and
%	\inst{2}%
%	DICE Group\\
%	Department of Computer Science\\
%	Paderborn University, Germany
%}

% One logo on title page
%\titlegraphic{\includegraphics[height=1cm]{DICE}}

% Two logos on title page
%\titlegraphic{
%	\includegraphics[height=1cm]{UPB}%
%	\hspace*{2cm}~%
%	\includegraphics[height=1cm]{DICE}%
%}



\begin{document}

% Title page without background
%{\usebackgroundtemplate{}
%\frame[plain]{\titlepage}}

% Title page with background
{\usebackgroundtemplate{\includegraphics[width=\paperwidth]{background-title-upb}
% German UPB logo
%\usebackgroundtemplate{\includegraphics[width=\paperwidth]{background-title-upb-german}
}
\frame[plain]{\titlepage}}

% Begin counting at second frame
\addtocounter{framenumber}{-1}



%\section{Motivation \& Contents}
\begin{frame}{Agenda}
%	\begin{columns}[T]
%	\column{0.45\textwidth}
%		\heading{Motivation}
%		After the title slide, typically a table of contents is presented. Make it more interesting by firstly \emph{introducing the problem} you are solving afterwards. That could also be on an own slide. Better than a long text like this is an image or keywords. Almost always.
%	\column{0.55\textwidth}
		\heading{Contents}\vspace*{.2cm}
		\hypersetup{linkcolor=textblue}
		\tableofcontents
%	\end{columns}
\end{frame}


\chapter{Introduction}
\label{ch:introduction}

This is the documentation of the \ac{template}.
In this chapter, you find everything you need to know to get started with your
thesis --- or at least the \LaTeX{} side of it. 
The other chapters of this document provide you with further details of the
template.
Besides a full documentation of the document class file that \ac{template}
uses and the auxiliary files \ac{template} supplies, we also provide you with
a short guide on certain aspects of \LaTeX{}, and particularly the \LaTeX{}
packages that \ac{template} loads by default.
However, it is assumed that you know some \LaTeX{} already.
If you are unfamiliar with \LaTeX{} you should consult a proper introduction
first, for example the \LaTeX{} guide by Oetiker \etal{} \cite{LatexGuide}.


\paragraph{Organization}
In the remainder of this chapter, we discuss how to
\hyperref[sec:introduction:start]{get started} with the \ac{template} template,
and what \hyperref[sec:introduction:folders]{directory structure} the template
provides and assumes.

In Chapter~\ref{ch:class_file}, we provide a documentation of the
\emph{upb\_cs\_thesis} document class provided by the \ac{template} template.
The template loads a number of \LaTeX{} packages by default.
Chapter~\ref{ch:packages} provides an overview.

Chapter~\ref{ch:guide} provides a guide to both beginners and experts on
aspects of packages that \ac{template} loads by default for you to use.
Particularly, the chapter discusses some style aspects of good writing,
references, including bibliographic references, graphics, and tables.
Finally, in Chapter~\ref{ch:makefiles}, we provide a documentation of and
troubleshooting guide for the \emph{Makefiles} that come with this template.


\input{chapters/introduction/start}
\input{chapters/introduction/folder_structure}

\chapter{The upb\_cs\_thesis Document Class}
\label{ch:class_file}

In this section we discuss what the \emph{upb\_cs\_thesis} class as used by the
\acs{template} template does, and what commands and environments it provides.
We go through the file in chunks of lines that belong together and provide
descriptions afterwards.



\begin{verbatim}
\NeedsTeXFormat{LaTeX2e}
\ProvidesClass{ubp_cs_thesis}[2017/12/18 initial version]
\end{verbatim}

\vspace*{-6pt} \noindent
The \acs{template} template is requires the \LaTeXe{} and provides the
\emph{upb\_cs\_thesis} documents class



\begin{verbatim}
\newif\ifgerman\germanfalse
\DeclareOption{german}{\germantrue}
\end{verbatim}

\vspace*{-6pt} \noindent
The class supports English and German language options.
While the German language option (lower case: \emph{german}!) must be
explicitly requested, the class defaults to English.



\begin{verbatim}
\ProcessOptions
\LoadClass[a4paper, 11pt, twoside, openright]{book}
\end{verbatim}

\vspace*{-6pt} \noindent
The class is based on the standard book class provided by \LaTeX{}.



\begin{verbatim}
\ifgerman
        \RequirePackage[ngerman]{babel}
\else
        \RequirePackage[english]{babel}
\fi
\end{verbatim}

\vspace*{-6pt} \noindent
The class loads language packs depending on the language options passed to the
class.


\begin{verbatim}
\RequirePackage[T1]{fontenc}
\RequirePackage[utf8]{inputenc}
\RequirePackage{lmodern}
\RequirePackage{geometry}
\RequirePackage{graphicx}
\RequirePackage{hyperref}
\RequirePackage{csquotes}
\RequirePackage{tikz}
\RequirePackage{calc}
\RequirePackage[explicit]{titlesec}
\end{verbatim}

\vspace*{-6pt} \noindent
A couple of standard packages are loaded, for details on the packages, see
Section~\ref{sec:packages:class_packages}.



\begin{verbatim}
\reversemarginpar
\geometry{a4paper, twoside, left=30mm, right=20mm, top=30mm, bottom=30mm}
\end{verbatim}

\vspace*{-6pt} \noindent
The general page layout has 3cm margins on the top and bottom of an A4 page.
The inner margin is 3cm wide, while the outer one is 2cm wide.
The first page is to be considered a right page, \ie{} inner margins on the
left hand side of the page.



\begin{verbatim}
\pagestyle{empty}
\pagenumbering{roman}
\end{verbatim}

\vspace*{-6pt} \noindent
The first few pages of a thesis are not to show any headers or page numbers.
This behavior is altered in the \verb+\tableofcontents+ command.
Until then, page numbers are counted in Roman numbers, although only displayed
on the table of contents page.



\begin{verbatim}
\newcommand{\@upbchaptername}{}
\newcommand{\@upbsectionname}{}
\newcommand{\ps@upb}{%
  \renewcommand{\@oddhead}{\leftmark\hfill}%
  \renewcommand{\@evenhead}{\hfill\rightmark}%
  \renewcommand{\@oddfoot}{\hfill\thepage\hfill}%
  \renewcommand{\@evenfoot}{\hfill\thepage\hfill}%
}
\end{verbatim}

\vspace*{-6pt} \noindent
This class provides the \emph{upb} page style that prints the page number
centered et the bottom of the page and chapter or section number and name on
odd and even pages, respectively.


\begin{verbatim}
\newcommand*{\emptypage}{\newpage\null\thispagestyle{empty}\newpage}
\end{verbatim}

\vspace*{-6pt} \noindent
The class provides the \verb+\emptypage+ command that takes no argument and
adds an empty page to the document.



\begin{verbatim}
\newcommand{\thetitle}{undefined}
\newcommand{\theauthor}{undefined}
\let\oldtitle=\title
\let\oldauthor=\author
\renewcommand{\title}[1]{\renewcommand{\thetitle}{#1}\oldtitle{#1}}
\renewcommand{\author}[1]{\renewcommand{\theauthor}{#1}\oldauthor{#1}}
\newcommand{\thethesistype}{undefined}
\newcommand{\thedegree}{undefined}
\newcommand{\theresearchgroup}{undefined}
\newcommand{\thesupervisor}{undefined}
\newcommand{\thesubmissiondate}{undefined}
\newcommand{\thesistype}[1]{\renewcommand{\thethesistype}{#1}}
\newcommand{\degree}[1]{\renewcommand{\thedegree}{#1}}
\newcommand{\researchgroup}[1]{\renewcommand{\theresearchgroup}{#1}}
\newcommand{\supervisor}[1]{\renewcommand{\thesupervisor}{#1}}
\newcommand{\submissiondate}[1]{\renewcommand{\thesubmissiondate}{#1}}
\end{verbatim}

\vspace*{-6pt} \noindent
The class provides seven commands for setting data for a thesis title page:
\begin{enumerate}
	\item author: \verb+\author+,
	\item title: \verb+\title+,
	\item thesis type: \verb+\thesistype+,
	\item academic degree: \verb+\degree+,
	\item supervisor's research group name: \verb+\researchgroup+,
	\item supervisor's name: \verb+\supervisor+, and
	\item thesis submission date: \verb+\submissiondate+.
\end{enumerate}
These values passed to these commands can be accessed via commands
\begin{enumerate}
	\item \verb+\theauthor+,
	\item \verb+\thetitle+,
	\item \verb+\thethesistype+,
	\item \verb+\thedegree+,
	\item \verb+\theresearchgroup+,
	\item \verb+\thesupervisor+, and
	\item \verb+\thesubmissiondate+, respectively.
\end{enumerate}
\noindent Since \verb+\author+ and \verb+\title+ commands are provided by
\LaTeX{} by default, we change their behavior slightly to make the values
passed to the commands available in a manner consistent with the other
commands, but we also preserve their original behavior.

These commands are used to typeset the title page as defined in
\mbox{pretext/titlepage.tex}



\begin{verbatim}
\newenvironment{abstract}{
    \begin{center}
        \begin{minipage}{.9\textwidth}
            \ifgerman
                \textbf{Zusammenfassung.} \hspace*{0.10pt}
            \else
                \textbf{Abstract.} \hspace*{0.10pt}
            \fi
}{
        \end{minipage}
    \end{center}
}
\end{verbatim}

\vspace*{-6pt} \noindent
The class provides the language sensitive \verb+abstract+ environment for
typesetting an abstract for your thesis.



\begin{verbatim}
\let\@upbtocold=\tableofcontents
\renewcommand{\tableofcontents}{
  \pagestyle{plain}	
  \@upbtocold
  \cleardoublepage
  \setcounter{page}{0}
  \pagestyle{upb}
  \pagenumbering{arabic}
  \renewcommand{\chaptermark}[1]{
    \markboth{\rm\chaptername\ \thechapter.\ #1}{}
  }
  \renewcommand{\sectionmark}[1]{
    \markright{\rm\thesection\ #1}{}
  }
}
\end{verbatim}

\vspace*{-6pt} \noindent
Starting with the table of contents, the thesis page numbers are displayed
(Roman numbers).
After the table of contents, page numbers are reset to 0, displayed in Arabic, 
and headers (lowercase upright chapter and section names) are shown. 
For this the behavior of \verb+\tableofcontents+ is altered, but its original
behavior is preserved.



\begin{verbatim}
\newlength{\@upbchapternumberwidth}
\newlength{\@upbtitlemaxwidth}
\newlength{\@upbtitletextwidth}
\end{verbatim}

\vspace*{-6pt} \noindent
These are some dimensions used by the upcoming commands to layout chapter
openings.



\begin{verbatim}
\titleformat{\chapter}{
  \normalfont\huge\bfseries
}{}{0em}{
  \newcommand*{\@upbprintscaledchapternumber}{
    \resizebox{!}{15mm}{\thechapter}
  }   
  \settowidth{\@upbchapternumberwidth}{
    \@upbprintscaledchapternumber
  }   
  \setlength{\@upbtitlemaxwidth}{
    .9\textwidth - \@upbchapternumberwidth 
  }   
  \settowidth{\@upbtitletextwidth}{#1}
  \flushright
  \rlap{
    \parbox[b]{\textwidth + 23mm}{
      \parbox[b]{\@upbtitlemaxwidth}{
        \ifdim\@upbtitletextwidth<\@upbtitlemaxwidth
          \hfill #1
	\else
          #1\hbadness=10000
        \fi 
      }   
      \hspace{12mm}
      \makebox[\@upbchapternumberwidth][b]{
        \raisebox{12mm}{\@upbprintscaledchapternumber}
      }   
      \hspace{-\@upbchapternumberwidth}
      \hspace{-12mm}
      \begin{tikzpicture}
        \fill[black] (0mm, 0mm) rectangle
           (\@upbchapternumberwidth +23mm, 10mm);
      \end{tikzpicture}
    }   
  }   
}
\end{verbatim}

\vspace*{-6pt} \noindent
These lines define the layout of a numbered chapter opening:
the chapter name to the left of a black box, the chapter number atop the bar.
The first few lines compute the width of the printed chapter number, and the
text available for the chapter name.
The width of the black box depend on the width of the chapter number, so we
account for arbitrary digit numbers.
If the chapter name is wide than the space available, line breaks are
automatically applied, the name becomes left aligned, and the base line of its
lowest line is the bottom line of the black box.
If there is sufficient space, the text is right aligned.



\begin{verbatim}
\titleformat{name=\chapter,numberless}{
  \normalfont\huge\bfseries
}{}{0em}{
  \settowidth{\@upbchapternumberwidth}{
    \resizebox{!}{15mm}{0}
  }   
  \setlength{\@upbtitlemaxwidth}{
    .9\textwidth - \@upbchapternumberwidth 
  }   
  \settowidth{\@upbtitletextwidth}{#1}
  \flushright
  \rlap{
    \parbox[b]{\textwidth + 23mm}{
      \parbox[b]{\@upbtitlemaxwidth}{
        \ifdim\@upbtitletextwidth<\@upbtitlemaxwidth
          \hfill
	\else
          #1  
        \fi 
      }   
      \hspace{8mm}
      \begin{tikzpicture}
        \fill[black] (0mm, 0mm) rectangle
            (\@upbchapternumberwidth +23mm, 10mm);
      \end{tikzpicture}
    }
  }
}
\end{verbatim}

\vspace*{-6pt} \noindent
This is a version of the chapter opening for unnumbered chapters.



\begin{verbatim}
\titlespacing*{\chapter}{0pt}{30pt}{25pt}
\end{verbatim}

\vspace*{-6pt} \noindent
This command sets the spacing around chapter openings; it is required by the
\emph{titlesec} package.

\usepackage{amsmath, amssymb, amsfonts}% mathematical symbols and the like
\usepackage{amsthm}% definitions, theorems, etc.
\usepackage[colorinlistoftodos]{todonotes}% marking open todos in text/on margins
\usepackage{subfig}% multi-part figures with separate captions per part
\usepackage{url}% render URLs correctly and make them clickable through the hyperref package
\usepackage{longtable}% tables that span multiple pages
\usepackage{booktabs}% tables that actually look good
\usepackage[nolist]{acronym}% consistently use acronyms

\usepackage{pgfgantt}

%Customizing GanttChart
\definecolor{barblue}{RGB}{153,204,254}
\definecolor{groupblue}{RGB}{51,102,254}
\definecolor{linkred}{RGB}{165,0,33}
\renewcommand\sfdefault{phv}
\renewcommand\mddefault{mc}
\renewcommand\bfdefault{bc}
\setganttlinklabel{s-s}{START-TO-START}
\setganttlinklabel{f-s}{FINISH-TO-START}
\setganttlinklabel{f-f}{FINISH-TO-FINISH}
\sffamily
\ganttset
{%
	y unit chart=0.8cm,
	canvas/.append style={fill=none, draw=black!35, line width=.5pt},
%	hgrid={*1{draw=black!35, line width=.1pt}},
	hgrid style/.style={draw=black!35,line width=.1pt},
	vgrid={*1{draw=black!35, line width=.1pt}},
	today rule/.style={draw=black!64,dash pattern=on 3.5pt off 4.5pt,line width=1.5pt},
	today label font=\small\bfseries,
	title/.style={draw=none, fill=none},
	title label font=\bfseries\footnotesize,
	title label node/.append style={below=7pt},
	progress label text={}, % \pgfmathprintnumber[precision=0, verbatim]{#1}\%
	include title in canvas=false,
	bar label font=\mdseries\small\color{black!70},
	bar label node/.append style={left=0cm},
	bar/.append style={draw=none, fill=black!63},
	bar incomplete/.append style={fill=barblue},
	bar progress label font=\mdseries\footnotesize\color{black!70},
	bar height=.5,
	bar top shift=0.1,
	group top shift=0.1,
	group incomplete/.append style={fill=groupblue},
	group left shift=0,
	group right shift=0,
	group height=.5,
	group peaks tip position=0,
	group label node/.append style={left=.6cm},
	group progress label font=\bfseries\small,
	link/.style={-latex, line width=1.5pt, linkred},
	link label font=\scriptsize\bfseries,
	link label node/.append style={below left=-2pt and 0pt}
}


\chapter{A Short Guide}
\label{ch:guide}
In this chapter, we discuss how to use the \ac{template}.
Although we mainly present information every versed \LaTeX{} user should know,
even an expert may gain some new knowledge.
Particularly, we discuss simple but important \hyperref[sec:guide:style]{style}
choices that you must make when writing \LaTeX{}.
We also discuss how to set clickable
\hyperref[sec:guide:references]{references} to different parts of your
thesis, such as chapters, figures, tables or full bibliographic references.
\hyperref[sec:guide:graphics]{Figures}, \hyperref[sec:guide:tables]{tables} and
\hyperref[sec:guide:bibliographies]{bibliographies} are also discussed in this
section. 

\section{Style}
\label{sec:guide:style}
This is just a simple guide on how to use the \ac{template}, so we do not get
into the details of how to write a good scientific text.
However, we give a short overview on the means that \LaTeX{} and \ac{template}
provide to make your life easier, and achieve your goal of writing a high
quality thesis.
In this section, we discuss proper use of \LaTeX{}'s math mode and the
footnotes.



\subsection{Math Mode}
\label{sec:guide:style:math}
Whenever you use the language of mathematics in your text, you should tell
\LaTeX{}!
Do so using the sequence \verb+\(+ at the start of the formula or
expression, and \verb+\)+ at the end.
These commands render the mathematical expression in-line.
For long, complex, or important expressions, \LaTeX{} offers the similar
\verb+\[+ and \verb+\]+ commands.\footnote{The use of \texttt{\$} and
\texttt{\$\$} is simply bad practice in \LaTeX{}. Using the same symbol as the
start and stop marking makes reading and understanding things really hard.
It also is confusing to some syntax highlighters in certain circumstances.
However, switching to math mode via \texttt{\$} may still be used in certain
contexts in which the \LaTeX{} equivalents do not work, for example in
\mbox{\texttt{\(\backslash\)item[\$foo\$]}} commands inside lists.} 

The various \emph{ams} packages leaded by default in file \mbox{packages.tex} provide
a number of environments that also switch to math mode, but introduce
additional functionality, like equation counting and multi-row formulas with
custom alignment of content. 

Whenever you use the language of mathematics in your text, you should tell
\LaTeX{}!
There really is a difference between x and \(x\)!
If you have a variable name that consists of multiple letters, you should
pass the name as an argument to command \verb+\mathit+, as to avoid confusion
between \(\mathit{abc}\) and \(abc\), where the latter is short for \(a \cdot b
\cdot c\).\footnote{If you haven't noticed, the spacing and letters are
slightly different.}

When you use your own mathematical operators, similar to \(\sin\), you should
explicitly tell \LaTeX{} that it is an operator.
Do so by using the \verb+\DeclareMathOperator+ command.
Command \verb+DeclareMathOperator+ works similar to \verb+\newcommand+, but
gets the spacing right; compare 
\[1 \testop 23 \quad \text{ and } \quad 1 \mathrm{top} 23,\] 
as produced by \verb+1 \testop 23+ and 
\verb+1 \mathrm{top} 23+.\footnote{\texttt{\(\backslash\)testop} was
defined via
\texttt{\(\backslash\)DeclareMathOperator\{\(\backslash\)testop\}\{top\}}.}



\subsection{Footnotes}
\label{sec:guide:style:footnotes}
In many cases, you may\footnote{read as: probably are} be inclined to use
footnotes to provide additional information via the \verb+\footnote+ command.
Don't! 
Footnotes break the flow of reading.
This means, you divert the reader's attention to some information you do not
consider to be sufficiently important to add it to the main text body.
As a result, using the footnote achieves the exact opposite of what it is
supposed to achieve.\footnote{That is: provide additional, less important
information that can be skipped without harm.}
You have probably experienced this yourself while reading this section.



\subsection{Acronyms}
\label{sec:guide:style:acronyms}
When writing your thesis, you may discover that some terms are repeated very
often.
As a result you want to abbreviate the terms.
This is normal and completely fine.
However, you have to take care that you do not start using an abbreviation
before you formally introduce it.
Otherwise your readers may be unable to understand your text.

Unfortunately, you are so used to the term and its abbreviation that you will
probably not notice that you use an abbreviation that has not been introduced
yet, particularly when you edit your text later on, \eg{} after proof reading.
Fortunately, there is a package to help you: the \emph{acronym} package that is
loaded by default via the \mbox{packages.tex} file.
\emph{acronym} allows you to define the term you use, the abbreviation and the
long form as \verb+\newacro{myacro}[MA]{my acronym}+,
where \verb+myacro+ is the term you use, \verb+MA+ is the abbreviation and
\verb+my acronym+ represents the long form.
You can insert the abbreviation called \verb+myacro+ by using
\verb+\ac{myacro}+, which is rendered as ``my acronym (MA)'' for its first
use, and as ``MA'' for each subsequent use.
The \emph{acronym} package provides variants of the \verb+\ac+ command to use
the plural form of the abbreviated term or to explicitly use the unabbreviated
form.


\section{References}
\label{sec:guide:references}
\LaTeX{} allows you to insert insert clickable references to other parts of the
document into your thesis.
These references help your readers avoid tedious scrolling to get to a specific
page or searching for some Figure~17, that can be located anywhere.

The clickable references are enabled from the \emph{hyperref} package that is
loaded by default via file \mbox{thesis.tex}. 
With \emph{hyperref} loaded, \LaTeX{} puts references into the table of
contents, so the reader can jump to the right section without a hassle.
However, it is good practice to insert such references yourself wherever
needed.

For example, your thesis's introduction will typically contain a paragraph that
provides an overview of the thesis's structure.
This is where you would typically add references to the sections that you talk
about, \eg{} the foundations of your work are presented in
Chapter~\ref{ch:packages}.
The reference to Chapter~\ref{ch:packages} is invoked by using command
\verb+\ref{ch:packages}+.
Of course, you need to tell \LaTeX{} what you mean by ``ch:packages.''
For that, you put a \verb+\label{ch:packages}+ right at the start of the
chapter called ``Packages and Commands,'' like you can see in the code for this
document, also replicated in Example~\ref{ex:code_label}.

\begin{example}
	\label{ex:code_label}
	This is the code for starting a new chapter and setting a label/jump
	mark:
	\begin{verbatim}
		\chapter{Packages and Commands}
		\label{ch:packages}
	\end{verbatim}
\end{example}
The \verb+\ref+ command is also used to create references to other parts of
your thesis, \eg{} figures, tables, equations, or the code example that you
find above. 

Typically, the references are rendered as numbers, like \ref{ex:code_label},
potentially including letters if the reference points to the appendix of your
document.
However, you can tell \LaTeX{} to make other text into a reference as well.
For example, the \hyperref[ex:code_label]{code example} above can also be
referenced via \verb+\hyperref[ex:code_label]{code example}+.\footnote{Note
that with \texttt{\(\backslash\)hyperref}, the label is the optional
argument!}
If you want to point to specific equations inside an equation environment, you
should use \verb+\eqref+ instead of \verb+\ref+, so the reference renders
correctly: surrounded by a pair of parentheses.

Two other classes of references are enabled by the packages \ac{template} loads
by default: URLs and bibliographic references.
The \emph{url} package in conjunction with \emph{hyperref} makes URL clickable,
and if your PDF viewer supports it, clicking such a URL accesses the URL using
your standard web browser.
Such references are created using the \verb+\url{URL}+ command.

\paragraph{Bibliographic references.}
Bibliographic references point to specific entries in the bibliography section
of your thesis, assuming you use the default setup of this template.
However, for bibliographic references you do not manually put jump marks/labels
into your document.
The marks are instead taken from the labels provided in the \mbox{.bib} file,
as discussed in Section~\ref{sec:guide:bibliographies}.
You can create bibliographic references using the \verb+\cite{label}+ command.
The \verb+\cite+ command correctly puts bibliographic references in square
brackets. 
Note that the \verb+\cite+ command can take an optional argument that is
typically used if you want to give further information, \eg{} if you talk about
Chapter~27 of the book referenced as \verb+FancyBookRef+, you should put the
reference as \verb+\cite[Chapter~27]{FancyBookRef}+.

\input{chapters/guide/graphics}
\section{Tables}
\label{sec:guide:tables}

Tables are a means to display a lot of information in a comprehensible manner.
\LaTeX{} offers many options to create and structure tables.
Unfortunately, given the options, many people have a tendency to put too many
structuring elements into their tables.
As a result, what is intended to improve the readers understanding has quite
the opposite effect.

Basic advice on what not to do in good tables typically includes:
\begin{itemize}
	\item never use vertical bars,
	\item use horizontal bars rarely, \eg{} for grouping multiple entries,
	and
	\item if there is a real possibility for readers to confuse entries from
	multiple rows as belonging together, use color coding (with decent(!)
	colors) to make boundaries clearer (there are \LaTeX{} packages for
	that, google them if necessary).
\end{itemize}
Following this advice, we present you with the basics on how to produce tables
that look good.

Tables are produced in a \emph{tabular} environment, such as in
Example~\ref{ex:table_simple}.
\begin{example}
\label{ex:table_simple}
This is a code example of a simple table.
\begin{verbatim}
\begin{tabular}{lcp{3cm}r}
  \toprule
  This & is & an & example \\
  \midrule
  Hello & World & I really have no idea what to write & here \\
  so & I just write & some nonsense & text. \\
  \bottomrule
\end{tabular}
\end{verbatim}
The code tells \LaTeX{} to produce a table (tabular environment) with four
columns (arguments \verb+l+, \verb+c+, \verb+p{3cm}+ and \verb+r+ to the
environment.
The four arguments state how to align the text in the respective column.
\begin{itemize}
	\item \verb+l+: Left aligned text; the column's width is automatically
	determined by the widest entry.
	\item \verb+c+: Center aligned text; the column's width is determined
	as for Option~\verb+l+.
	\item \verb+p{3cm}+: Block aligned text; the column's with is set
	manually to 3cm.
	\item \verb+r+: Right aligned text; the column's width is determined as
	for Option~\verb+l+.
\end{itemize}
Inside the environment, we start with a \verb+\toprule+, a thick-ish horizontal
bar to indicate the beginning of the table.
The header of the table follows, the change of columns indicated by the
\verb+&+ character; the row is finished with the \verb+\\+ mark.
The end of the header section is visually marked by another (thin) bar created
by \verb+\midrule+.
In the example, two more rows of contents follow, each ended by the \verb+\\+
mark, again with the \verb+&+ character to separate cells.
The table's end is visually marked by another thick-ish horizontal bar.

The table is rendered as follows:

\begin{tabular}{lcp{3cm}r}
	\toprule
	This & is & an & example \\
	\midrule
	Hello & World & I really have no idea what to write & here \\
	so & I just write & some nonsense & text. \\
	\bottomrule
\end{tabular}
\end{example}

Note that the commands \verb+\toprule+, \verb+\midrule+ and \verb+\bottomrule+
are provided by the \emph{booktabs} package that is loaded from file
\mbox{packages.tex}. 
Without these, \LaTeX{}'s \verb+\hline+ can be used, but messes up the spacing
between rows, resulting in ugly tables.


Typically, you do not want to surround your tables with context, \eg{} add a
caption to it and be able to point to it from any part of the document.
You can achieve this by using the \emph{table} floating environment as shown in
Example~\ref{ex:table_env}.
As a floating environment, it may be placed where it fits, rather than where
(relative to the source code) it is defined.
\begin{example}
\label{ex:table_env}
An example of the \emph{table} environment.
\begin{verbatim}
\begin{table}[tbhp]
  \caption{Some descriptive text}
  \label{tab:name}
  \begin{tabular}{...}
    ...
  \end{tabular}
\end{table}
\end{verbatim}
The optional parameter passed to the environment determines its preferred
placement.
The order of the letters gives the precedence.
In this case, placement of the table at the top (\verb+t+) of a page is
preferred, followed by a placement on the bottom (\verb+b+) of a page.
If neither is possible, \LaTeX{} is to attempt to put the table where it is
defined (\verb+h+, relative to the source code), and if everything else fails,
the table will be put onto a special page (\verb+p+) that may only hold
floating environments. 
This last option is one of the reasons, why the caption should be some
descriptive text that a reader can understand without having read the main
text.

Inside the environment, you find the \verb+caption+ of the table, followed by
its \verb+label+.
The label does not work properly if it precedes the caption.
Finally, the table environment holds the relevant table environment.
\emph{Note} that, unlike with figures, with tables, the caption precedes the
table!
\end{example}

Sometimes you may have tables that are too long to fin onto a single page.
This is problematic, since \LaTeX{} does not put page breaks into tables.
The \emph{longtable} package comes to rescue.
It provides the \emph{longtable} environment that puts page breaks into tables,
repeats the table header automatically on each page that holds a portion of the
table, and indicates that the table is continued on the next page if necessary.
Example~\ref{ex:table_long} shows how to create long tables.
\emph{Note} that with long tables, you do not need/want a floating environment.
As compensation the \verb+\caption+ and \verb+label+ commands can be used with
the \emph{longtable} environment directly.
\begin{example}
\label{ex:table_long}
This is a code example of a long table
\begin{verbatim}
\begin{longtable}{l}
  \caption{Some descriptive text}
  \label{tab:name} \\

  \toprule
  HEADER \\
  \midrule
  \endfirsthead

  \midrule
  SECOND_HEADER
  \midrule
  \endhead

  \midrule
  FOOTER
  \endfoot

  \bottomrule
  \endlastfoot

  Table contents
\end{longtable}
\end{verbatim}

The \emph{longtable} environment takes a parameter just like the \emph{tabular}
environment that tells \LaTeX{} about the number of columns end text alignment.
The \verb+\caption+ and \verb+\label+ commands follow as in the \emph{table}
environment.
This part is ended with the \verb+\\+ mark.

What follows is a definition of the first header \verb+HEADER+ of the table,
its end is marked by \verb+\endfirsthead+.
The secondary header \verb+SECOND_HEADER+ is the header that gets repeated
after each page break.
The end of its definition is marked by \verb+\endhead+.
The footer \verb+FOOTER+ gets repeated just before every page break that occurs
in the table.
Its definition is ended by \verb+\endfoot+
The footer that marks the end of the table gets defined last (in this case a
simple \verb+\bottomrule+).
The end of its definition is marked by \verb+\endlastfoot+.
Finally, the tables contents are put into \emph{longtable} environment, just
like you would put in \emph{inside} the \emph{tabular} environment.
\end{example}

\section{Bibliographies}
\label{sec:guide:bibliographies}

Typically you do not want to manage you bibliographic references yourself.
\LaTeX{} and its helper program \emph{bibtex} manage them for you.
You just need to tell them what references you are going to use.
You do that by putting a bibliographic reference --- called \emph{bib entry}
--- into file \mbox{bibliography.bib}. 

The bib entries are as shown in Example~\ref{ex:bib_entry}.
The key=value pairs depend on the type of the bib entry.
Certain keys must be present in an entry of a given type, while others can be
missing.
\begin{example}
\label{ex:bib_entry}
This is an example bib entry of type ``inproceedings.'' It starts with an @
symbol, immediately followed by its type and an openeing curly brace and a
label (LABEL in this example) and a comma.
What follows is a comma-separated list of key=value pairs, with the values
being surrounded by curly braces.
The bib entry closes with a final curly brace.
\begin{verbatim}
@inproceedings{LABEL,
  author    = {First Author and Second Author and ...},
  editor    = {A List of Fancy People},
  title     = {Example Title},
  booktitle = {Proceedings of Conference 1468},
  pages     = {123--456},
  publisher = {Publisher's Name},
  year      = {1468},
}
\end{verbatim}
Note that you can enforce a certain type of letter capitalization, \ie{} all
caps, by putting the respective term inside an extra pair of curly braces,
\eg{} \verb+publisher = {ACM},+ may be rendered as Acm, while 
\verb+publiser = {{ACM}},+ is always rendered as ACM.
\end{example}

Using \emph{bibtex}, you can compile a nicely looking list of bibliographic
references.
The list is included via the \emph{biblatex} package at the place indicated by
the \verb+\printbibliography+ command.
You can refer to the individual list entries by using the label (in the example
LABEL) of the corresponding bib entry in a \verb+\cite+ command as described in
Section~\ref{sec:guide:references}.
Hence, it is useful to derive a bib entry's label from the information provided
by the entry itself, \eg{} the title.

While \emph{bibtex} provides means to manage your bibliographic references,
they do not help you with obtaining all relevant, and particularly complete,
bib entries. 
You should create bib entries yourself, unless all other options fail.
Typically, publishers of computer science literature provide bib entries on
their web pages.
Also Google Scholar\footnote{\url{https://scholar.google.com}} provides bib
entries.
However, both sources sometimes provide very incomplete bib entries.
Complete entries can typically obtained from
DBLP.\footnote{\url{http://dblp.org/search/index.php}}



\chapter{Makefiles}
\label{ch:makefiles}

This \acs{template} template comes with two Makefiles to make your life easier.
One of the files is located inside the template's main folder, the other one is
located in the figures sub-folder.
The main Makefile can be used to compile the thesis's source files into a PDF
file, while the figures directory's Makefile is used for creating graphics out
of certain types of raw data.
In this chapter, we discuss what both files do.

You can compile your thesis without these Makefiles.
However, the Makefiles make sure the compilation process happens in the right
order.
For a simple compilation, type ``make'' into your console and hit Enter.
This starts the simple compilation process (calling pdflatex twice).
Type ``make complete'' and hit Enter, in order to compile the source files,
and include compiled images and bibliographic references.
This latter option may pose problems in very specific circumstances.
Therefore we provide a section on troubleshooting next.



\section{Makefile Troubleshooting}
\label{sec:makefiles:trouble}
The Makefiles introduce certain dependencies, \ie{} applications they need to
properly function.
However, they will typically not pose a problem.
The issues that may occur are:
\begin{enumerate}
	\item your thesis's bibliography file is empty, or
	\item dependencies are not installed, but relevant files exist.
\end{enumerate}

\paragraph{Empty bibliography file.}
The compilation process may fail if the \mbox{bibliography.bib} file does not
contain a reference. In this case, either add a reference, or change Line~8 of
the main Makefile from  \verb+complete: images lit short+ to 
\verb+complete: images short+, or comment out Lines~12 and 13 of the main
Makefile by prepending them with a hashtag (\#).
The first option is to be preferred if you already know at least one
bibliographic reference that you will use, the second option is reasonable if
you will not have any bibliographic reference (very unlikely), and the third
option is more or less a quick and dirty hack.

\paragraph{Missing dependencies.}
The compilation process may also fail, if you have \mbox{.svg} files in your
figures directory, but \emph{inkscape} is not installed, or you have
\mbox{.tex} files in the figures directory but either \emph{pdflatex} or
\emph{pdfcrop} is not installed.
In either case, you can install the relevant dependencies, or remove the
problematic files, or comment out the problematic lines of the figures
directory's Makefile by prepending them with a hashtag (\#).
Which lines to comment out depend on the file type that causes the problems;
see Section~\ref{sec:makefiles:figures} for details.




\section{Main Makefile}
\label{sec:makefiles:main}
The main Makefile is designed to compile the \LaTeX{} source files of your
thesis into the final PDF file.
The Makefile provides two main recipes for obtaining the PDF, a short and a
long version.
The short version is invoked via ``make'' or ``make short,'' the long
version is called via ``make complete.''
 
We now have a look at the Makefile itself.
The first line stores the file name (without the file extension) of your
thesis's main \LaTeX{} file in variable \texttt{THESIS}
Typically, this is \mbox{thesis.tex}, and THESIS store the value ``thesis,''
accordingly.

Lines~4 to 6 are the recipe for the short compilation process, they simply call
pdflatex twice on your thesis's main source file.

Line~8 contains the recipe for the long compilation process. 
However, rather than containing the process itself, it depends on three other
recipes that are called automatically: ``images,'' ``lit'' and ``short.''
Hence (see below and above) the long compilation process created image files
from raw data in the figures directory, processes bibliographic references and
finally puts the PDF together.

Lines~10 to 12 process bibliographic references.
First, \emph{pdflatex} is invoked on your thesis's main file in order to
collect the bibliographic references that you actually use, and then 
\emph{bibtex} for processing bibliographic references is called.

Lines~14 to 15 compile raw image data from the figures directory.
The do so by changing into the directory and executing the directory's
Makefile; for details, see the next \hyperref[sec:makefiles:figures]{section}.



\section{figures's Makefile}
\label{sec:makefiles:figures}
The figures sub-folder's Makefile is designed to process raw data and transform
them into useful images.
It handles vector graphics in the \mbox{.svg} format, as well as graphics
described via tikz\footnote{\TeX{} Ist Kein Zeichenprogramm, a \LaTeX{}
package.} in \mbox{.tex} files.
The reasoning behind putting tikz graphics in separate files is that compiling 
the code into graphics is somewhat time consuming, so recompiling graphics that
do not change should be avoided.

We now go through the Makefile and describe its purposes.
The first line collects all \mbox{.svg} files in the ``figures'' directory, and
stores their file names in variable \texttt{SVG\_FILES}.
The second line creates a variable \texttt{INKSCAPE\_FILES} and initializes it
with all the file names stored in \texttt{SVG\_FILES} with the \mbox{.svg} file
extension replaced by \mbox{.pdf\_tex}.
Similarly, Lines~4 and 5 collect the \mbox{.tex} files and corresponding
\mbox{.pdf} files in variables \texttt{TEX\_FILES} and \texttt{TIKZ\_FILES}.

Line~7 says that, by default, we want to create the \mbox{.pdf\_tex} files from
\mbox{.svg} files, and \mbox{.pdf} files from \mbox{.tex} files.
The following lines say how this transformation process works.
However, the transformation is only applied to source files that have changed
recently, \ie{} if the target file does not exist yet, or source file's last
modified date is newer than the target file's last modified date.

Lines~9 and 10 are the recipe for creating \mbox{.pdf\_tex} files out of
\mbox{.svg} files.
For this, the application \emph{inkscape} is used.
The recipe calls inkscape in command line mode and asks it to export the source
file's drawing and text as two separate files.
The first output file is a \mbox{.pdf} and contains the drawing itself.
The second file is the desired \mbox{.pdf\_tex} that is a \LaTeX{} file in
disguise.\footnote{The \mbox{.pdf\_tex} file extension helps with avoiding
confusion with other \mbox{.tex} files.}
It contains \LaTeX{} code that puts embeds the \mbox{.pdf} file into your
thesis, but also contains the text from the \mbox{.svg} file and overlays the
\mbox{.pdf} graphic with the text.
As a result, the text uses the same font as your document, and all text from
the \mbox{.svg} file gets interpreted as \LaTeX{} code.
\emph{Note} that due to some bug in the inkscape export, you may observe
\LaTeX{} errors during compilation if your \mbox{.pdf} file contains more than
a single page, \ie{} if the \mbox{.svg} file was created with inkscape in the
first place, you must put all graphic elements onto the same layer before
exporting.

Lines~12 to 14 are the recipe for getting \mbox{.pdf} files from \mbox{.tex}
files.
For that, the source file is compiled using pdflatex, as expected.
Afterwards, the resulting \mbox{.pdf} file is cropped to the drawing area, with
some surrounding margins.
This adds the applications \emph{pdflatex} and \emph{pdfcrop} as dependencies.






%%%%%%%%%%%%%%%%%%%%%%%%%%%%%%%%%
%\begin{frame}
%	
%\end{frame}
%
%%\section{Quick Start}
%\begin{frame}[fragile]{Quick Start}
%	Create your first slide:
%	\begin{enumerate}
%		\item Copy all \emph{*.sty} files into a directory
%		\item Copy \emph{packages.tex} into the directory
%		\item Create a .tex file and add the code listed below
%		\item Generate your slide using LaTeX
%	\end{enumerate}
%	\lstset{language=TeX, escapechar=\@, keywordstyle=\color{templatetext}, numbers=none, emph={claw,packages,tex}, emphstyle=\color{textblue}}
%	\begin{lstlisting}[caption={Minimal Example}]
%\documentclass{beamer}
%\usetheme{claw}
%\usepackage{amsmath, amssymb, amsfonts}% mathematical symbols and the like
\usepackage{amsthm}% definitions, theorems, etc.
\usepackage[colorinlistoftodos]{todonotes}% marking open todos in text/on margins
\usepackage{subfig}% multi-part figures with separate captions per part
\usepackage{url}% render URLs correctly and make them clickable through the hyperref package
\usepackage{longtable}% tables that span multiple pages
\usepackage{booktabs}% tables that actually look good
\usepackage[nolist]{acronym}% consistently use acronyms

\usepackage{pgfgantt}

%Customizing GanttChart
\definecolor{barblue}{RGB}{153,204,254}
\definecolor{groupblue}{RGB}{51,102,254}
\definecolor{linkred}{RGB}{165,0,33}
\renewcommand\sfdefault{phv}
\renewcommand\mddefault{mc}
\renewcommand\bfdefault{bc}
\setganttlinklabel{s-s}{START-TO-START}
\setganttlinklabel{f-s}{FINISH-TO-START}
\setganttlinklabel{f-f}{FINISH-TO-FINISH}
\sffamily
\ganttset
{%
	y unit chart=0.8cm,
	canvas/.append style={fill=none, draw=black!35, line width=.5pt},
%	hgrid={*1{draw=black!35, line width=.1pt}},
	hgrid style/.style={draw=black!35,line width=.1pt},
	vgrid={*1{draw=black!35, line width=.1pt}},
	today rule/.style={draw=black!64,dash pattern=on 3.5pt off 4.5pt,line width=1.5pt},
	today label font=\small\bfseries,
	title/.style={draw=none, fill=none},
	title label font=\bfseries\footnotesize,
	title label node/.append style={below=7pt},
	progress label text={}, % \pgfmathprintnumber[precision=0, verbatim]{#1}\%
	include title in canvas=false,
	bar label font=\mdseries\small\color{black!70},
	bar label node/.append style={left=0cm},
	bar/.append style={draw=none, fill=black!63},
	bar incomplete/.append style={fill=barblue},
	bar progress label font=\mdseries\footnotesize\color{black!70},
	bar height=.5,
	bar top shift=0.1,
	group top shift=0.1,
	group incomplete/.append style={fill=groupblue},
	group left shift=0,
	group right shift=0,
	group height=.5,
	group peaks tip position=0,
	group label node/.append style={left=.6cm},
	group progress label font=\bfseries\small,
	link/.style={-latex, line width=1.5pt, linkred},
	link label font=\scriptsize\bfseries,
	link label node/.append style={below left=-2pt and 0pt}
}


%\begin{document}
%\begin{frame} Hello World @\textbackslash@end{frame}
%@\textbackslash@end{document}
%	\end{lstlisting}
%\end{frame}
%
%%\section{Text Formatting}
%%\subsection{Predefined Styles}
%\begin{frame}{Text Formatting}{Predefined Styles}
%	\begin{itemize}
%		\item You could \emph{emphasize} important parts \\(Maybe distinguish between \bad{problems} and \good{solutions})
%		\item Use alert to display \alert{warnings}
%		\item Use the url command (\url{https://dice-research.org/}) or the href command (\href{https://dice-research.org/}{DICE}) for links
%		\item Highlight ``\term{predefined terms}'' like brands and\\ \tech{TechnicalTerms} like software components
%	\end{itemize}
%\end{frame}
%
%%\subsection{Additional Commands}
%\begin{frame}{Text Formatting}{Additional Commands}
%	Use combinations for other concepts:
%	\begin{itemize}
%		\item Text styles: \textbf{bold}, \textit{italic}, \underline{underlined}, \textsc{small caps}
%		\item Font families: \texttt{monospaced}, \textsf{sans serif}, \textrm{roman}
%		\item Text colors:
%			\bluedark{bluedark},
%			\gray{gray},
%			\magenta{magenta},
%			\blue{blue},
%			\orange{orange},
%			\purple{purple},
%			\red{red},
%			\turquoise{turquoise},
%			\green{green}
%		\item Text sizes: {\tiny tiny}, {\scriptsize scriptsize}, {\footnotesize footnotesize}, {\small small}, {\normalsize normalsize}, {\large large}, {\Large Large}, {\LARGE LARGE}, {\huge huge}, {\Huge Huge}
%	\end{itemize}
%\end{frame}
%
%%\section{Code Listings \& Frame Arguments}
%\begin{frame}[fragile]{Code Listings \& Frame Arguments}
%	Use these arguments to configure frames:
%	\setbeamercolor{local structure}{fg=textpurple}
%	\begin{description}
%		\item[\texttt{fragile}\hfill]
%		Specially interpreted contents, e.g. for listings
%		\item[\texttt{plain}\hfill]
%		No headlines, footlines, sidebars; e.g. for large images\\
%		\gray{To also remove background images use:
%		\texttt{\{\textbackslash usebackgroundtemplate\{\}[...]\}}}
%		\item[\texttt{squeeze}\hfill]
%		Squeezes vertical spaces, e.g. for long contents
%		\item[\texttt{shrink}\hfill]
%		Shrinks frame, e.g. for long contents
%	\end{description}
%	\lstset{language=TeX, escapechar=\@, morekeywords={begin,end,frame}, emph={fragile}}
%	\begin{lstlisting}[caption={Frame Options}]
%\begin{frame}[fragile]{Code Listings \& Frame Arguments}
%	% [...]
%@\textbackslash@end{frame}
%	\end{lstlisting}
%\end{frame}
%
%%\section{Mathematics \& Miscellaneous}
%\begin{frame}{Mathematics \& Miscellaneous}
%	\begin{itemize}
%		\item Math\footnote[frame]{This is a footnote also working in columns}: $5^{2}=3^{2}+4^{2}$
%		\item Equations:
%	\end{itemize}
%	\begin{equation}
%		\sum_{n = 1}^{\infty} \frac{1}{n} = 1 + \frac{1}{2} + \frac{1}{3} + \frac{1}{4} + \frac{1}{5} + \dots 
%	\end{equation}
%\end{frame}
%
%%\section{Blocks}
%\begin{frame}[shrink]{Blocks}
%	\begin{block}{This is a Block}
%		\begin{itemize}
%			\item This is an item
%		\end{itemize}
%		\begin{enumerate}
%			\item This is enumeration item
%		\end{enumerate}
%	\end{block}
%	\begin{exampleblock}{This is an Example Block}
%		\begin{itemize}
%			\item This is an item
%		\end{itemize}
%		\begin{enumerate}
%			\item This is enumeration item
%		\end{enumerate}
%	\end{exampleblock}
%	\begin{alertblock}{This is an Alert Block}
%		\begin{itemize}
%			\item This is an item
%		\end{itemize}
%		\begin{enumerate}
%			\item This is enumeration item
%		\end{enumerate}
%	\end{alertblock}
%\end{frame}
%
%% Examples for alternative colors. Configure this in packages.tex, not here.
%\setbeamercolor{block title}{bg=lightorange,fg=templatetext}
%\setbeamercolor{block body}{bg=backgroundorange}
%\setbeamercolor{block title example}{bg=lightgreen,fg=templatetext}
%\setbeamercolor{block body example}{bg=backgroundgreen}
%\setbeamercolor{block title alerted}{bg=lightred,fg=templatetext}
%\setbeamercolor{block body alerted}{bg=backgroundred}
%\begin{frame}[shrink]{Blocks}
%	\begin{block}{This is a Block}
%		\begin{itemize}
%			\item This is an item
%		\end{itemize}
%		\begin{enumerate}
%			\item This is enumeration item
%		\end{enumerate}
%	\end{block}
%	\begin{exampleblock}{This is an Example Block}
%		\begin{itemize}
%			\item This is an item
%		\end{itemize}
%		\begin{enumerate}
%			\item This is enumeration item
%		\end{enumerate}
%	\end{exampleblock}
%	\begin{alertblock}{This is an Alert Block}
%		\begin{itemize}
%			\item This is an item
%		\end{itemize}
%		\begin{enumerate}
%			\item This is enumeration item
%		\end{enumerate}
%	\end{alertblock}
%\end{frame}
%
%%\section{Tables}
%\begin{frame}{Tables}
%	\begin{table}[]
%	\centering
%	\renewcommand{\arraystretch}{1.1}
%	\begin{tabular}{ll}
%	\hline
%	\textbf{Topic} & \textbf{Content} \\ \hline
%	Generator & Use tools like \href{https://www.tablesgenerator.com/}{tablesgenerator.com} \\
%	Large tables & Try the frame option \texttt{[shrink=.8]} \\
%	 & (center table with \texttt{{\textbackslash}hspace*\{5cm\}})\\
%	Large tables & Combine the \href{https://ctan.org/pkg/longtable}{longtable} package and \\
%	 & the frame option \texttt{[allowframebreaks]}\\
%	Style & Try the \href{https://ctan.org/pkg/booktabs}{booktabs} package \\ \hline
%	\end{tabular}
%	\end{table}
%\end{frame}
%
%%\section{Graphs}
%\begin{frame}{Graphs}
%	\centering
%	\begin{tikzpicture}[node distance=1cm and 2cm] 
%		\node[resource] (pb) {Subject}; 
%		\node[literal, right=of pb] (code) {Object};
%		\draw[->] (pb) -- node[above]{Predicate} (code);
%	\end{tikzpicture}
%\end{frame}
%
%\begin{frame}{Thank you!}
%	\centering
%	\vspace{2.5cm}
%	{\huge\textbf{Questions?}}
%	\\\vspace{2.5cm}
%	{\large Data Science Group at Paderborn University}
%	\\\vspace{.1cm}
%	\centering
%	\begin{tabular}{rl}
%	Web: & \href{https://dice-research.org/}{dice-research.org} \\
%	Code: & \href{https://github.com/dice-group}{github.com/dice-group} \\
%	Twitter: & \href{https://twitter.com/DiceResearch}{@DiceResearch}
%	\end{tabular}
%\end{frame}
%
%\section{Appendix \& References}
%\appendix 
%
%\begin{frame}[t,allowframebreaks]{References}
%	\scriptsize
%	\nocite{*}
%	\bibliographystyle{ieeetr}
%	\bibliography{bibliography}
%\end{frame}
%
%\begin{frame}{Appendix}{Predefined Base Colors}
%	\begin{itemize}
%		\item {\color{primarybluedark}\rule{.7cm}{.4cm} primarybluedark}
%		\item {\color{primarybluelight}\rule{.7cm}{.4cm} primarybluelight}
%		\item {\color{primarygraylight}\rule{.7cm}{.4cm} primarygraylight}
%		\item {\color{primarygraydark}\rule{.7cm}{.4cm} primarygraydark}
%		\item {\color{secondarymagenta}\rule{.7cm}{.4cm} secondarymagenta}
%		\item {\color{secondaryblue}\rule{.7cm}{.4cm} secondaryblue}
%		\item {\color{secondarygreen}\rule{.7cm}{.4cm} secondarygreen}
%		\item {\color{secondaryorange}\rule{.7cm}{.4cm} secondaryorange}
%		\item {\color{secondarypurple}\rule{.7cm}{.4cm} secondarypurple}
%		\item {\color{activeyellow}\rule{.7cm}{.4cm} activeyellow}
%		\item {\color{activered}\rule{.7cm}{.4cm} activered}
%		\item {\color{activeturquoise}\rule{.7cm}{.4cm} activeturquoise}
%		\item {\color{activegreen}\rule{.7cm}{.4cm} activegreen}
%		\item {\color{specificblue}\rule{.7cm}{.4cm} specificblue}
%	\end{itemize}
%\end{frame}
%
%\begin{frame}{Appendix}{Predefined Text Colors}
%	\begin{itemize}
%		\item {\color{textbluedark}\rule{.7cm}{.4cm} textdarkblue}
%		\item {\color{textgray}\rule{.7cm}{.4cm} textgray}
%		\item {\color{textmagenta}\rule{.7cm}{.4cm} textmagenta}
%		\item {\color{textblue}\rule{.7cm}{.4cm} textblue}
%		\item {\color{textorange}\rule{.7cm}{.4cm} textorange}
%		\item {\color{textpurple}\rule{.7cm}{.4cm} textpurple}
%		\item {\color{textred}\rule{.7cm}{.4cm} textred}
%		\item {\color{textturquoise}\rule{.7cm}{.4cm} textturquoise}
%		\item {\color{textgreen}\rule{.7cm}{.4cm} textgreen}
%		\item {\color{textbluespecific}\rule{.7cm}{.4cm} textbluespecific}
%	\end{itemize}
%\end{frame}
%
%\begin{frame}{Appendix}{Predefined Element Colors}
%	\begin{itemize}
%		\item {\color{elementgray}\rule{.7cm}{.4cm} elementgray}
%		\item {\color{elementmagenta}\rule{.7cm}{.4cm} elementmagenta}
%		\item {\color{elementblue}\rule{.7cm}{.4cm} elementblue}
%		\item {\color{elementorange}\rule{.7cm}{.4cm} elementorange}
%		\item {\color{elementpurple}\rule{.7cm}{.4cm} elementpurple}
%		\item {\color{elementyellow}\rule{.7cm}{.4cm} elementyellow}
%		\item {\color{elementred}\rule{.7cm}{.4cm} elementred}
%		\item {\color{elementturquoise}\rule{.7cm}{.4cm} elementturquoise}
%		\item {\color{elementgreen}\rule{.7cm}{.4cm} elementgreen}
%		\item {\color{elementbluespecific}\rule{.7cm}{.4cm} elementbluespecific}
%	\end{itemize}
%\end{frame}
%
%\begin{frame}{Appendix}{Predefined Light Colors}
%	\begin{itemize}
%		\item {\color{lightgray}\rule{.7cm}{.4cm} lightgray}
%		\item {\color{lightmagenta}\rule{.7cm}{.4cm} lightmagenta}
%		\item {\color{lightblue}\rule{.7cm}{.4cm} lightblue}
%		\item {\color{lightorange}\rule{.7cm}{.4cm} lightorange}
%		\item {\color{lightpurple}\rule{.7cm}{.4cm} lightpurple}
%		\item {\color{lightyellow}\rule{.7cm}{.4cm} lightyellow}
%		\item {\color{lightred}\rule{.7cm}{.4cm} lightred}
%		\item {\color{lightturquoise}\rule{.7cm}{.4cm} lightturquoise}
%		\item {\color{lightgreen}\rule{.7cm}{.4cm} lightgreen}
%		\item {\color{lightbluespecific}\rule{.7cm}{.4cm} lightbluespecific}
%	\end{itemize}
%\end{frame}
%
%\begin{frame}{Appendix}{Predefined Background Colors}
%	\begin{itemize}
%		\item {\color{backgroundgray}\rule{.7cm}{.4cm}} backgroundgray
%		\item {\color{backgroundmagenta}\rule{.7cm}{.4cm}} backgroundmagenta
%		\item {\color{backgroundblue}\rule{.7cm}{.4cm}} backgroundblue
%		\item {\color{backgroundorange}\rule{.7cm}{.4cm}} backgroundorange
%		\item {\color{backgroundpurple}\rule{.7cm}{.4cm}} backgroundpurple
%		\item {\color{backgroundyellow}\rule{.7cm}{.4cm}} backgroundyellow
%		\item {\color{backgroundred}\rule{.7cm}{.4cm}} backgroundred
%		\item {\color{backgroundturquoise}\rule{.7cm}{.4cm}} backgroundturquoise
%		\item {\color{backgroundgreen}\rule{.7cm}{.4cm}} backgroundgreen
%		\item {\color{backgroundbluespecific}\rule{.7cm}{.4cm}} backgroundbluespecific
%	\end{itemize}
%\end{frame}

\end{document}