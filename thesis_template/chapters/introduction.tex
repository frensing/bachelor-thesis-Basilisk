\chapter{Introduction}
\label{ch:introduction}

This is the documentation of the \ac{template}.
In this chapter, you find everything you need to know to get started with your
thesis --- or at least the \LaTeX{} side of it. 
The other chapters of this document provide you with further details of the
template.
Besides a full documentation of the document class file that \ac{template}
uses and the auxiliary files \ac{template} supplies, we also provide you with
a short guide on certain aspects of \LaTeX{}, and particularly the \LaTeX{}
packages that \ac{template} loads by default.
However, it is assumed that you know some \LaTeX{} already.
If you are unfamiliar with \LaTeX{} you should consult a proper introduction
first, for example the \LaTeX{} guide by Oetiker \etal{} \cite{LatexGuide}.


\paragraph{Organization}
In the remainder of this chapter, we discuss how to
\hyperref[sec:introduction:start]{get started} with the \ac{template} template,
and what \hyperref[sec:introduction:folders]{directory structure} the template
provides and assumes.

In Chapter~\ref{ch:class_file}, we provide a documentation of the
\emph{upb\_cs\_thesis} document class provided by the \ac{template} template.
The template loads a number of \LaTeX{} packages by default.
Chapter~\ref{ch:packages} provides an overview.

Chapter~\ref{ch:guide} provides a guide to both beginners and experts on
aspects of packages that \ac{template} loads by default for you to use.
Particularly, the chapter discusses some style aspects of good writing,
references, including bibliographic references, graphics, and tables.
Finally, in Chapter~\ref{ch:makefiles}, we provide a documentation of and
troubleshooting guide for the \emph{Makefiles} that come with this template.


\section{Getting Started}
\label{sec:introduction:start}

In this section, we briefly discuss how to set up your thesis to work with the
\acs{template} template.
This is a four step program:
\begin{enumerate}
	\item choosing a language,
	\item setting up the title page of your thesis,
	\item writing its abstract, and 
	\item writing the thesis.
\end{enumerate}

The \acs{template} template can set up your thesis to use either English or
German language options.
By default, English is used.
If you want to use German, edit the line of \emph{thesis.tex} that reads
\verb+\documentclass[]{upb\_cs\_thesis}+ and write \emph{german} (lower case!)
into the pair of brackets.

What you need to set up the title page are a \emph{title}, your \emph{own
name}, the \emph{type} of the thesis and the academic \emph{degree} you aim
for, the name of your \emph{supervisor} and his or her \emph{research group's
name}, as well as a submission date.
Setting up your thesis's title page, by editing the main file
\emph{thesis.tex}. 
In the section marked ``your thesis title, [\dots],'' you find certain
commands to which you pass the information listed above.
Pass your thesis's title to the \verb+\thesis+ command, and your own name to
the \verb+\author+ command.
The type of your thesis is passed to the \verb+\thesistype+ command.
You can comment in or out any of the example types given in the file, or pass
a string of your own choice.
Similarly, for the academic degree choose any of the given examples or pass
your own choice.
Your supervisor's research group's name is passed to the \verb+\researchgroup+
command, while his or her name goes into the \verb+\supervisor+ command.
The submission date of your thesis is passed to the \verb+\submissiondate+
command; you can pass the current date via the \verb+\today{}+ command, \ie{}
not changing the template file.

If you already know your thesis's abstract, write it down in file
\mbox{abstract.tex}.
You can leave the file unchanged, in which case this template documentation's
abstract is used instead until you change file \mbox{abstract.tex}.

Finally, start writing your thesis.
Your texts should go into the chapters directory, as described in
Section~\ref{sec:introduction:folders:chapters}.


\section{Organisation: Directories}
\label{sec:introduction:folders}

The \acs{template} template assumes a simple directory structure.
It has a main directory that holds all files relevant to the template itself.
Three complementary directories hold your texts, figures and appendices.



\subsection{Main Directory}
\label{sec:introduction:folders:main}
Your main directory contains the four sub-directories
\begin{enumerate}
	\item \texttt{chapters},
	\item \texttt{figures},
	\item \texttt{pretext}, and
	\item \texttt{appendices}.
\end{enumerate}
These directories are further described in separate sections. The main
directory also contains a couple of files: 
\begin{description}
	\item[\texttt{abstract.tex}] This file holds the abstract of your
	thesis.

	\item[\texttt{appendix.tex}] This file groups your appendices together.
	The file is \verb+\input+ into the main file \mbox{thesis.tex}.
	The appendices themselves should go into the \texttt{appendices}
	folder, and be included into \mbox{appendix.tex} via the \verb+\input+
	command.

	\item[\texttt{body.tex}] This file contains the main text of your
	thesis.
	The file is \verb+\input+ into the main file \mbox{thesis.tex}.
	The texts themselves should go into the \texttt{chapters} folder, and be
	included into \mbox{body.tex} via the \verb+\input+ command.

	\item[\texttt{commands.tex}] Your custom \LaTeX{} commands should go
	into this file.
	The file is \verb+\input+ into the preamble of the main file
	\mbox{thesis.tex}.
	By default, \mbox{commands.tex} defines commands for common
	abbreviations, and some environments.
	See Section~\ref{sec:packages:commands} for further information.

	\item[\texttt{literature.bib}] Your bibliographic references should go
	into this file.
	See Section~\ref{sec:guide:bibliographies} for further information.

	\item[\texttt{Makefile}] This file can make your life easier. It
	provides means for simple and complex compilation processes. For
	details, see Section~\ref{sec:makefiles:main}.

	\item[\texttt{packages.tex}] This is where you include \LaTeX{}
	packages.
	The file is \verb+\input+ as the first file into the preamble of the
	main file \mbox{thesis.tex}.
	By default, it loads the auxiliary packages as described in
	Section~\ref{sec:packages:auxiliary_packages}.

	\item[\texttt{thesis.tex}] This is the main file for your thesis.
	It defines the document class, includes the preamble, and loads files
	\mbox{body.tex} and \mbox{appendix.tex}.
	It also sets up the title page, legalities, and the table of contents.

	\item[\texttt{upb\_cs\_thesis.cls}] This is defines the document class
	used by the \ac{template} template.
	See Chapter~\ref{ch:class_file} for details.
\end{description}



\subsection{Chapters Directory}
\label{sec:introduction:folders:chapters}
This directory is where the \mbox{.tex} files of your thesis go.
It is advisable to have one file per chapter.
For long chapters, you should create additional files for smaller bits of text,
\eg{} on the section or even subsection level.
The low level files should then be \verb+\input+ into the respective chapter's
\mbox{.tex} file, while the chapter's \mbox{.tex} file should be \verb+input+
in main directory's \mbox{body.tex}.



\subsection{Figures Directory}
\label{sec:introduction:folders:figures}
This directory is where your graphics go.
See Section~\ref{sec:guide:graphics} on how to include graphics in your thesis.
By default, this directory contains the university's logo
(\texttt{uni-logo.pdf}), a \texttt{Makefile}, and the \texttt{template\_files}
sub-directory. 

The university's logo is used on the title page.
The \emph{Makefile} of the \emph{figures} directory converts some types of raw
data into graphics; see Section~\ref{sec:makefiles:figures} for details.
Finally, the \emph{template\_files} sub-directory contains the
\texttt{tikz\_template.tex} file, that can be used to create
tikz\footnote{\TeX{} Ist Kein Zeichenprogramm, a \LaTeX{} package for
programming graphics} graphics that the figures directory's \emph{Makefile} can
properly process.



\subsection{Pretext Directory}
\label{sec:introduction:folders:pretext}
This directory contains two files \texttt{erklaerung.tex} and
\texttt{titlepage.tex}.
The former contains a legal statement, the latter defines the layout of your
titlepage.
Typically, you do not need to edit these files.
You fill in the details of your title page as described in
Section~\ref{sec:introduction:start}.



\subsection{Appendices Directory}
\label{sec:introduction:folders:appendices}
This directory is where your thesis's appendices go.
The files should be \verb+\input+ into your thesis via the \mbox{appendix.tex}
file.
By default, this directory is empty.

