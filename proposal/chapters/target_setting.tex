\chapter{Formulation of Target Setting}
\label{ch:target_setting}

The target of this thesis is to setup and deploy an instance of the Basilisk service platform.
Since Basilisk is not yet fully developed, the missing parts will be designed and implemented first, before the system can be deployed publicly. \\

The platform will be deployed on a virtual machine provided by the IRB-department of the Paderborn University.
The hardware requirements are 8-16 CPU-cores, 256GB of RAM, and 4TB of Storage for different datasets and benchmarks. \\

Different benchmarks will be performed on the \tsp{} under test.
Synthetic benchmarks will be generated with WatDiv\cite{alucDiversifiedStressTesting2014}.
FEASIBLE\cite{saleemFEASIBLEFeatureBasedSPARQL2015} will be used to generate benchmarks from real-world datasets of varied sizes and structures.
Real-world datasets that will be used are a subset of the WikiData and DBpedia.
For these datasets there are query logs contained in LSQ\cite{saleemLSQLinkedSPARQL2015} which will be provided to the FEASIBLE framework.
Other benchmarks that will be available on the platform are LUBM\cite{guoLUBMBenchmarkOWL2005} and SP²Bench\cite{schmidtSP2BenchSPARQLPerformance2008}.\\


The metrics that are calculated after each benchmark, are the ones provided by the IGUANA framework \cite{MetricsIguanaDocumentation}\cite{conradsIguanaGenericFramework2017}, since the IGUANA framework is used in the Basilisk platform to perform the benchmarks.
These metrics provided can be seen in table \ref{table:iguana_metrics}:

\begin{table}[tbhp]
	\begin{tabular}{lp{10cm}}
		\toprule
		Name            & Description                                                  \\ \midrule
		NoQ             & number of successfully executed Queries                      \\
		QMPH            & number of executed Query Mixes per Hour                      \\
		NoQPH           & NoQ per hour                                                 \\
		QPS             & queries per second                                           \\
		penalized QPS   & failed queries get a penalty that is included in this metric \\
		AvgQPS          & average QPS                                                  \\
		penalizedAvgQPS & AvgQPS with penalty included                                 \\ \bottomrule
	\end{tabular}
	\caption{Metric provided by the IGUANA framework\cite{MetricsIguanaDocumentation}.}
	\label{table:iguana_metrics}
\end{table}


At least two \tsp{} will be evaluated with the framework of which one should be Tentris\cite{bigerlTentrisTensorBasedTriple2020}.
The goal of this use-case is to assess how well the system helps automating the execution and evaluation of benchmarks.
This includes examining which steps could be further automated and which interfaces could be added to get specific data.

If unforeseen problems arise, these will be documented and evaluated, to learn for further attempts.


%Beim use-case steht weniger die Auswertung der Benchmarks im Mittelpunkt, sondern in wie weit die Ausführung und Auswertung durch das durch das System automatisiert wird. 
%Dabei kannst du dann z.B. schauen: welche Schritte ließen sich noch weiter automatisieren, fehlen noch relevante schnittstellen um an bestimmte daten zu kommen, etc.



%---
%
%Goals of the thesis:
%- Whats the problem with existing solutions? with the stuff you explained in the motivation
%- What's the goal of the thesis in a nutshell? solve that problem / analyze the solution / compare potential solutions / ...
%- Then go into detail. What are the specifics of what you are going to do? Explain intermediate goals
%- Explain how your evaluation will look like. Describe your test environment. 
%- Provide optional goals
%
%
%Evaluation
%- was macht basilisk was vorher nicht da war
%- versch systeme mit selben zweck vergleichen
%- suche? eigentlich niht verfügbar
%- richtung: einfaches seetting
%- benchmark
%- 3 triplestores
%- wie lange mit system / Ohne system
%- schnell im basilisk zw benchmarks wechseln
%- im voraus planen was zu testen