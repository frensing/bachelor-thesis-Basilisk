\chapter{Formulation of Target Setting}
\label{ch:target_setting}

The target of this thesis is to setup and deploy an instance of the Basilisk service platform.
Since Basilisk is not yet fully developed, the missing parts will be designed and implemented first, before the system can be deployed publicly.
After that a use-case with 3 benchmarks will be executed and evaluated.
At least two \tsp{} will be evaluated with the framework of which one should be Tentris\cite{tentris}.
The goal of this use-case is to assess how well the system helps automating the execution and evaluation of benchmarks.
During the evaluation of the use-case

If unforeseen problems arise, these will be documented and evaluated, to learn for further attempts.


Beim use-case steht weniger die Auswertung der Benchmarks im Mittelpunkt, sondern in wie weit die Ausführung und Auswertung durch das durch das System automatisiert wird. 
Dabei kannst du dann z.B. schauen: welche Schritte ließen sich noch weiter automatisieren, fehlen noch relevante schnittstellen um an bestimmte daten zu kommen, etc.



%---
%
%Goals of the thesis:
%- Whats the problem with existing solutions? with the stuff you explained in the motivation
%- What's the goal of the thesis in a nutshell? solve that problem / analyze the solution / compare potential solutions / ...
%- Then go into detail. What are the specifics of what you are going to do? Explain intermediate goals
%- Explain how your evaluation will look like. Describe your test environment. 
%- Provide optional goals
%
%
%Evaluation
%- was macht basilisk was vorher nicht da war
%- versch systeme mit selben zweck vergleichen
%- suche? eigentlich niht verfügbar
%- richtung: einfaches seetting
%- benchmark
%- 3 triplestores
%- wie lange mit system / Ohne system
%- schnell im basilisk zw benchmarks wechseln
%- im voraus planen was zu testen