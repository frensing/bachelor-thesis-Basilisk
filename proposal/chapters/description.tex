\chapter{Description and Motivation}
\label{ch:description}

In the field of Semantic Web, knowledge graphs are an important structure to represent data and its relationships.
To easily store and query the data in these knowledge graphs, some data structure or database is needed.
The special kind of database developed to store knowledge graphs are called \tsp{}. \\

Since knowledge graphs can contain huge amounts of data which can also be subject to many changes, \tsp{} need to be able to handle many different workloads.
Some scenarios need to handle huge amount of data being added, while others need to handle a lot of changes on the current data.
To better test and compare \tsp{} in these diverse scenarios, benchmarks are performed to allow an appropriate comparison between different \tsp{}\cite{saleemHowRepresentativeSPARQL2019}.

Benchmarks in general are used to measure and compare the performance of computer programs and systems with a defined set of operations.
Often they are designed to mimic and reproduce a particular type of workload to the system.
In the context of \tsp{}, a benchmark usually consists of creating a given knowledge graph on which multiple queries and operations are performed.

Often \tsp{} are developed in long iterations and are bench-marked only in a late stage of such an development iteration.
Today benchmarks and the evaluation of their results are usually done manually and bind developers time.
Thus, performance regressions are  found very late or never.


Several benchmarks for \tsp{} have been proposed \cite{saleemHowRepresentativeSPARQL2019}.
\iguana{} is a benchmark-independent execution framework \cite{conradsIguanaGenericFramework2017} that can measure the performance of \tsp{} under several parallel query request.
Currently the benchmark execution framework needs to be installed and benchmarks need to be started manually.
Basilisk is a continuous benchmarking service for \tsp{} which internally uses \iguana{} to perform the benchmarks.
The idea is that the Basilisk service will check automatically for new versions of \tsp{} and start benchmarks with the \iguana{} framework.
Further it should be possible to start custom benchmarks on demand.
If a new version is found in a provided GitHub- or DockerHub-repository, Basilisk will automatically setup a benchmark environment and starts a benchmarking suite.

This means that developers do not have to worry about performing benchmarks at different stages of development.



%---
%
%- What general topic are we interested in
%- why is this topic interesting in general
%- where is it used
%- why does it matter
%- who implements it
%- what problem does it solve?
%
%- What is your personal motivation to deal with this topic?
%- Which interesting problems do you expect?
%
%Explain some more background for the topic, help the reader, who may have never heard of the topic, understand what you are talking about.
%
%Current state of the research:
%- What is a trivial way to solve x?
%- How is x usually solved in practice?
%- what are typical ingredients / techniques to construct the systems we are interested in?
%- How does ? solve the problem?
%- What does the algo in ? do roughly?
%
%
%------
%
%The topic I have to offer has the working title is "Basilisk -- Continuous Benchmarking for Triplestores". At the core, it is developing and deploying a CI/CD tool that hooks into github and/or docker registries.
%
%Described in more detail:
%
%Triplestores -- the database backend of knowledge graphs -- are
%typically developed in long iterations and are bench-marked -- if at
%all -- only in a very late stage of such an iteration. Benchmarking
%and evaluation of benchmarking results are typically done manually
%and binds developer's time. Thus, performance regressions are found
%very late or never.
%
%With Basilisk we started to develop a continuous benchmarking
%platform for triplestore which hooks into github and docker image
%registries.
%
%On events like pull requests or newly published versions of
%triplestores, a benchmarking suite is run automatically.
%
%The first version of Basilisk
%(https://github.com/dice-group/Basilisk,
%https://github.com/dice-group/basilisk-frontend) is already
%implemented. It is based on the benchmarking tool IGUANA
%(https://github.com/dice-group/IGUANA) and Docker. (It requires
%triple stores to be dockerized).
%
%The thesis task is to:
%
%
%
%The thesis can be extended to a paper in a scientific journal (e.g.
%ISWC) on the resource track with you as first author. 
%
