\chapter{Description}
\label{ch:description}

\tsp{} are the database backend of knowledge graphs. 
They are needed to easily store and query data of knowledge graphs used in the Semantic Web.
Since knowledge graphs can contain huge amount of data and can also be subject to many changes, \tsp{} need to be able to handle many different workloads.
To better test and compare \tsp{} in these diverse scenarios, benchmarks are performed to allow an appropriate comparison between different \tsp{}.

Benchmarks in general are used to measure and compare the performance of computer programs and systems with a defined set of operations.
Often they are designed to mimic and reproduce a particular type of workload to the system.
In the context of \tsp{}, a benchmark usually consists of creating a big knowledge graph and performing multiple queries and operations on the data.

Several benchmarks for \tsp{} have been proposed.
\textsc{Iguana} is a benchmark-independent execution framework \cite{IGUANA} that can measure the performance of \tsp{} under several parallel query request.
Currently the benchmark execution framework needs to be installed and benchmarks need to be started manually.
Basilisk is a continuous benchmarking service for \tsp{} which internally uses \textsc{Iguana} to perform the benchmarks.
The idea is that the Basilisk service will check automatically for new versions of \tsp{} and start benchmarks with the \iguana{} framework.
Further it should be possible to start custom benchmarks on demand.



---

- Write some words in general about the topic you are going to tackle in your thesis. Motivate why is this topic interesting in general, where is it used, who implements it what problem does it solve?

- What is your personal motivation to deal with this topic?
- Which interesting problems do you expect?

What general topic are we interested in, why does it matter, where is it used, what problem does it solve?
Explain some more background for the topic, help the reader, who may have never heard of the topic, understand what you are talking about.

Current state of the research:
- What is a trivial way to solve x?
- How is x usually solved in practice?
- what are typical ingredients / techniques to construct the systems we are interested in?
- How does ? solve the problem?
- What does the algo in ? do roughly?


Benchmarking erklären?
\aclp*{ts} erklären motivierend


- halbe Seite

