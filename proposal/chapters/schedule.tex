\chapter{Schedule}
\label{ch:schedule}

The Schedule for the proposed task in Chapter \ref{ch:tasks} is shown in figure \ref{fig:schedule}.
The last point "Buffer" is meant as a buffer for unforeseeable events that could alter the planned time schedule.

\begin{figure}[tbph]
	\begin{ganttchart}
		[%today=0, %"TODAY" vertical line
%		x unit=0.9cm, %extend sheets to 1cm
		]{1}{20}
		\gantttitle[title label node/.append style={below left=7pt and -3pt}]{Weeks:\quad1}{1}
		\gantttitlelist{2,...,20}{1} \\
		\ganttgroup[progress=0]{Preparation}{1}{6} \\
		\ganttbar[progress=0]{Software architecture analysis and review}{1}{3} \\
		\ganttbar[progress=0]{Solution Design}{4}{6} \\
		
		\ganttgroup[progress=0]{Implementation}{7}{12} \\
		\ganttbar[progress=0]{Solution Implementation}{7}{10} \\
		\ganttbar[progress=0]{Deployment of Basilisk and frontend}{11}{12} \\
		
		\ganttgroup[progress=0]{Evaluation}{13}{18} \\
		\ganttbar[progress=0]{Perform benchmarks for \tsp{}}{13}{15} \\
		\ganttbar[progress=0]{Fix critical bugs, document non-critical}{16}{18}\\
		
		\ganttgroup[progress=0]{Thesis Writing}{1}{20} \\
		\ganttbar[progress=0]{Related Work \& Background}{1}{4} \\
		\ganttbar[progress=0]{Approach}{5}{7} \\
		\ganttbar[progress=0]{Implementation}{7}{12} \\
		\ganttbar[progress=0]{Evaluation}{13}{18} \\
		\ganttbar[progress=0]{Summary and Discussion}{16}{18} \\
		\ganttbar[progress=0]{Buffer}{19}{20}\\
	\end{ganttchart}
	
	\caption{Proposed schedule for the work on the thesis.}
	\label{fig:schedule}
\end{figure}

%---
%
%21 weeks  of work
%Gantt chart: pgfgantt package