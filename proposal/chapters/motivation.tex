\chapter{Motivation}
\label{ch:motivation}

Triplestores are the database backend of knowledge graphs. 
They are typically developed in long iterations and are bench-marked, if at all, in a very late stage of such an development iteration.
Typically Benchmarking and evaluation of the results are done manually and bind developers time.
Thus, performance regressions are found very late or never.

---


The topic I have to offer has the working title is "Basilisk -- Continuous Benchmarking for Triplestores". At the core, it is developing and deploying a CI/CD tool that hooks into github and/or docker registries.

Described in more detail:

Triplestores -- the database backend of knowledge graphs -- are
typically developed in long iterations and are bench-marked -- if at
all -- only in a very late stage of such an iteration. Benchmarking
and evaluation of benchmarking results are typically done manually
and binds developer's time. Thus, performance regressions are found
very late or never.

With Basilisk we started to develop a continuous benchmarking
platform for triplestore which hooks into github and docker image
registries.

On events like pull requests or newly published versions of
triplestores, a benchmarking suite is run automatically.

The first version of Basilisk
(https://github.com/dice-group/Basilisk,
https://github.com/dice-group/basilisk-frontend) is already
implemented. It is based on the benchmarking tool IGUANA
(https://github.com/dice-group/IGUANA) and Docker. (It requires
triple stores to be dockerized).

The thesis task is to:



The thesis can be extended to a paper in a scientific journal (e.g.
ISWC) on the resource track with you as first author. 