\section{Architecture Review}
\label{sec:architecture_review}
In this section we review the architecture of the three services of the Basilisk platform.
We point out possible problems with current implementations and list missing implementations that need to be added.


\subsection{Management of Repositories and Configurations}
\label{sec:management_repo_config}
Currently the observed repositories are managed and stored in the \ac{hcs} while the configurations for the \tsp{} are managed and stored in the \ac{jms}.
This makes it difficult to internally link a repository to a \ts{} configuration, since they are stored in different services.

The current implementations tries to solve this problem, by sending events about repository creations from the \ac{hcs} to the \ac{jms}.
This results in the duplication of the repository storage in both services.
This contradicts the idea of microservice, which should be separated as much as possible from each other.

Therefore we recommend restructuring the management of repositories.


\subsection{Code Refactoring}
\label{sec:code_refactor}
During the code analysis some inconsistencies in the code style and duplicate code snippets have been found.
In other parts the code structured differs to the design patterns recommended for the Spring and Spring Boot framework.

In general an in-depth code refactoring is recommended to increase readability and maintainability of the source code. 


\subsection{Missing Implementations}
\label{sec:review_missing_impl}
The Basilisk platform is not yet fully implemented.
After reviewing the source code the following overview was created:

\subsubsection{\acl{hcs}}
The implementation of the \acl{hcs} is quite complete.
Only small additions have to be implemented.

\begin{itemize}
	\item The REST endpoints for deleting \gh{} and \dockh{} repositories needs to be added.
	
	\item Currently Pull Requests in \gh{} repositories can not be observed.
\end{itemize}


\subsubsection{\acl{jms}}
The implementation of the \acl{jms} is mainly missing the REST API and some internal logic.
The following REST endpoints have to be added:

\begin{itemize}
	\item Adding / removing \ts{} configurations
	
	\item Adding / removing dataset configurations
	
	\item Adding / removing query configurations
\end{itemize}

The structure for the internal data models has logical errors and is in parts incomplete.
The resulting database model for the internal database is contradicting itself on some points and needs a restructure.
\\

Since the \ac{jms} also manages the running and pending benchmark jobs, the REST API and internal logic for managing these jobs needs to be implemented too.

\begin{itemize}
	\item List running / pending jobs and their status
	
	\item Set the status for a job
\end{itemize}



\subsubsection{\acl{tbs}}
The implementation of the \acl{tbs} is currently only a bare structure of basic classes.
Big parts of the logic still needs to be implemented.

The existing classes are mainly for storing data models, configurations and basic message queue interactions.
These classes do not carry much functionality.

The main functionality needs to be implemented.
This consists of setting up the Docker containers which contain the \tsp{} for benchmarking:

\begin{itemize}
	\item Pulling Code from \gh{}
	\item Pulling images form \dockh{}
	
	\item building Docker containers from Dockerfiles / images
	
	\item connection to the Docker containers
\end{itemize}

Then the usage of the \iguana{} framework needs to be implemented.
The framework needs to be setup to write the benchmark results to the \acl{jsts}.
\\

To have a better control of the running jobs and the benchmarking service in general we recommend to add a small REST API.
This API could be similar to the one of the \ac{hcs}, that starts and stops the continuous checking.
The API for the \ac{tbs} can function like a switch, which indicates if a new benchmarking job will be started or not.
If it is set to off, the current benchmarking job will be finished, but no new job will be started.

Lastly, after the benchmark, some cleanup of the Docker containers needs to be implemented.

\subsubsection{User Management and Security}
\label{sec:review_user_management}
Currently the REST APIs of the services allow interactions with any user.
If the platform is needed to run publicly, some user management and further security measures are needed.
That includes registering new users and user groups with different user rights.
Some users should only be able to read benchmark results, while other should be able to create repositories and abort jobs.

Secondly, confidential information need to be kept secret.
This is for example the OAuth-Key needed for accessing private \gh{} repositories.


\subsection{Frontend}
\label{sec:review_frontend}
\todo[inline]{Was in dieser Section schreiben?}
The frontend introduces a new programming languages and frameworks.
A short review of the current source code resulted in the following findings:
\begin{itemize}
	\item Currently only a small web view is implemented
	\item Functionality to communicate with the REST APIs is missing
\end{itemize}


As explained in section \ref{sec:time_schedule} the priority for the frontend has been lowered.
The focus for the thesis lies in finishing the main services and their functionality.

