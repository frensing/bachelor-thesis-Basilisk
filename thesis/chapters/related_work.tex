\chapter{Related Work}
\label{ch:related_work}
%%%%%%%%%%%%%%%%%%
% - description of existing techniques and approaches
%%%%%%%%%%%%%%%%%%

This chapter reviews the state of the art of \ts{} benchmarking.\\

Several benchmarks have been proposed and developed.
Many of these existing benchmarks focus on different goals and scenarios to test the \tsp{}.
Benchmarking in general is explained in section \ref{sec:benchmark}.

\section{Synthetic Benchmarks}
\label{sec:synthetic_benchmarks}
The LUBM Benchmark\cite{guoLUBMBenchmarkOWL2005} is a synthetic benchmark which focuses on the reasoning and inferencing capabilities of the \tsp{} under test.
The test data is about the university domain and can be generated to arbitrary size.
The benchmark provides fourteen extensional queries that represent and test a variety of properties.

Another synthetic benchmark is SP²Bench\cite{schmidtSP2BenchSPARQLPerformance2008}.
The data generated stems from the DBLP scenario. The benchmark generation tries to accomplish that the key characteristics and word distributions are close to the original DBLP dataset.
The provided queries are mostly complex and the mean size of the result sets is above one million\cite{saleemFEASIBLEFeatureBasedSPARQL2015}.
They also test for SPARQL features like union and optional graph patterns.

The WatDiv suite generates a synthetic benchmarks and consists of multiple tools\cite{alucDiversifiedStressTesting2014}.
First the data generator which generates scalable and customizable datasets based on the WatDiv data model schema.
The query template generator generates diverse query templates which will then be used to generate actual queries.
The queries get generate with the query generator which instantiates the templates with actual RDF terms from the generated dataset.
For each template multiple queries can be generated.
The benchmark only focuses on SELECT queries that does not make use of Union and Optional patterns.

\section{Benchmarks Using Real Data}
\label{sec:benchmarks_real_data}
FEASIBLE is a benchmark generation framework which generates datasets and queries from provided query logs\cite{saleemFEASIBLEFeatureBasedSPARQL2015}.
This has the advantage that the data used for the benchmark could stem from queries about a specialized real world topics rather than an abstract synthetic model.
FEASIBLE can also generate queries for the other SPARQL query types beside SELECT.

\section{Benchmark Execution Frameworks}
% Most benchmarks come with their own execution framework \todo{cite!}

\todo{fill}

\subsection{IGUANA}
\label{sec:iguana}
IGUANA is a SPARQL benchmark execution framework\cite{conradsIguanaGenericFramework2017}.
The framework uses the SPARQL endpoint of the \ts{} under test to load, update and query the data.
It allows the measurement of the performance during loading and updating of data as well as parallel requests to the \ts{}.
IGUANA is independent of any benchmarks which allows it to run in different configurations and with different existing benchmarks and datasets.

\todo{setup / benchmark execution}




\subsection{Hobbit Framework}
Hobbit framework 
\todo{Fill}
