\section{Software Development}
The following topics can be grouped under the field of software development.
For the topic of benchmarks (section \ref{sec:benchmark}) we focus on database benchmarks and especially \ts{} benchmarks, since this is the main goal of the Basilisk platform.
The sections Microservice and Microservice Architecture (\ref{sec:microservice}, \ref{sec:microservice_architecture}) explain the basic idea and concept of the microservice architecture style.
In the sections RabbitMQ and Spring (\ref{sec:rabbitmq}, \ref{sec:spring}) we give a short introduction and description of the main technologies that are used for the development of the Basilisk platform.

\subsection{Benchmark}
\label{sec:benchmark}
Benchmarks for databases consist of a data set and a set of operations or queries which will be performed on the data set.
These operations are designed to simulate a particular type of workload to the system.
The goal of a benchmark is to measure different metrics for a better comparison between various systems.
Metrics used for databases and \tsp{} are \eg, number of executed queries and queries per second\cite{IguanaDocumentationMetrics}.

A distinction is made between micro and macro benchmarks.
Micro benchmarks focus on testing the performance of single components of a system.
Macro benchmarks test the performance of a system as a whole.
The benchmarks performed by the Basilisk platform, which will be set up in this thesis, will only perform macro benchmarks.

\subsection{Microservice}
\label{sec:microservice}
A microservice is an independently deployable piece of software that only implements functionalities that are closely related to the main task of the service \cite{dragoniMicroservicesYesterdayToday2017}.
All Microservices can be individually deployed and managed and they interact via messages through a defined protocol with other services.
The idea is that individual microservices can be combined like modules to create any desired complex software.

\subsection{Microservice Architecture}
\label{sec:microservice_architecture}
A microservice architecture is a way of designing a software application as a set of microservices which interact with each other to provide the designed functionality \cite{dragoniMicroservicesYesterdayToday2017, MicroservicesHttpsMartinfowler}.
The functionality of the application gets split up into microservices which interact only through a defined message protocol.
This allows for a distributed system in which the individual service could be implemented in different programming languages and also could be located on different servers.


\subsection{RabbitMQ}
\label{sec:rabbitmq}
RabbitMQ is an open-source message broker that supports different messaging protocols like MQTT, STOMP and AMQP.
The system supports a variety of asynchronous messaging techniques \eg, delivery acknowledgment, flexible routing\cite{RabbitMQWebsiteHttps}.

In the context of the Basilisk platform we only need the most basic functionalities of message queues with a single producer and a single consumer.
Since RabbitMQ is a widely used message broker, the Spring framework (\ref{sec:spring}) already comes with the needed libraries to work with the RabbitMQ system.


\subsection{Spring and Spring Boot}
\label{sec:spring}
Spring\footnote{\url{https://spring.io/}} is a widespread open-source Java framework which facilitates the development process for various kinds of java applications and systems.

Spring Boot\footnote{\url{https://spring.io/projects/spring-boot}} is an extension to the Spring framework that follow the convention-over-configuration design paradigm.
This means that the implementation of applications has to follow common design conventions that replace a need for configuration files for many standard scenarios.
Spring Boot also comes with preconfigured standard libraries for the Spring platform to ease the development for many standard applications like web-apps or microservices.

Spring and Spring Boot come with different annotations to decorate classes and methods that configure these automatically and tell the Spring framework how to handle and interact with their objects.

The Spring framework and Spring Boot use different software design conventions to structure the code and classes.
The package structure found in the source code of the Basilisk microservices is influenced by Spring Boot and tries to represent those different design conventions.


\subsection{Software Design Patterns}
\label{sec:software_design_patterns}


\subsubsection{Repository Pattern}
\todo{fill}

\subsubsection{Domain Driven Design}
\todo{fill}

\subsection{Stateful / Stateless Microservices}
\todo{fill}