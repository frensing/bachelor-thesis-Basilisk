\section{Software Development}
\label{sec:software_dev}
The following topics can be grouped under the field of software development.
For the topic of benchmarks (section \ref{sec:benchmark}) we focus on database benchmarks and especially \ts{} benchmarks since this is the main task of the Basilisk platform.
The sections Microservice (\ref{sec:microservice}) and Microservice Architecture (\ref{sec:microservice_architecture}) explain the basic idea and concept of the microservice architecture.
In the sections RabbitMQ and Spring (\ref{sec:rabbitmq}, \ref{sec:spring}), we give a short introduction and description of the main technologies that are used for the development of the Basilisk platform.

\subsection{Benchmark}
\label{sec:benchmark}
Benchmarks for databases consist of a dataset and a set of operations or queries which will be performed on the dataset.
These operations are designed to simulate a particular workload in the system.
The goal of a benchmark is to measure different metrics for a better comparison between various systems.
Metrics used for databases and \tsp{} are \eg, the number of executed queries (NoQ), queries per second (QPS), and query mixes per hour (QMPH) \cite{IguanaDocsMetrics}.

A distinction is made between micro and macro benchmarks \cite{gibbsMicrobenchmarksVsMacrobenchmarks}.
Micro benchmarks focus on testing the performance of single components of a system.
Macro benchmarks test the performance of a system as a whole.
The benchmarks performed by the Basilisk platform will only cover macro benchmarks.

\subsection{Microservice}
\label{sec:microservice}
A microservice is an independently deployable piece of software that only implements functionalities that are closely related to the main task of the service \cite{dragoniMicroservicesYesterdayToday2017}.
All Microservices can be individually deployed and managed.
They interact via messages through a defined protocol with other services.
The idea is that individual microservices can be combined like modules to create any desired complex software.

A typical example is an online shop system that is divided into microservices.
One service of this system could be managing the customer data, like contact details and shipping addresses.
If another service needs this information, it can send a request to this service over defined protocols.


\subsection{Microservice Architecture}
\label{sec:microservice_architecture}
Microservice architecture is a way of designing a software application as a set of microservices that interact with each other to provide the designed functionality \cite{dragoniMicroservicesYesterdayToday2017, MicroservicesHttpsMartinfowler}.
The application's functionality gets split up into microservices that interact only through a defined message protocol.
This allows for a distributed system in which the individual services could be implemented in different programming languages and also could be located on different servers.

Extending the example from section \ref{sec:microservice} we have a microservice architecture with three services.
One service manages the customer data, a second service manages the incoming orders, and the last service manages the shipping process.
When an order arrives in the order service it will send a message to the shipping service, with the order items.
The shipping service will then request the customer's shipping address from the customer-data service.


\subsection{Spring and Spring Boot}
\label{sec:spring}
Spring\footnote{\url{https://spring.io/}} is a widespread open-source Java framework that facilitates the development process for various kinds of java applications and systems.

Spring Boot\footnote{\url{https://spring.io/projects/spring-boot}} is an extension to the Spring framework that follows the convention-over-configuration design paradigm.
This means that the implementation of applications has to follow standard design conventions that replace the need for configuration files for many standard scenarios.
Spring Boot also comes with pre-configured standard libraries for the Spring platform to ease the development of many standard applications like web apps or microservices.

Spring and Spring Boot use different annotations to decorate classes and methods.
These annotations configure the classes automatically and tell the Spring framework how to handle and interact with their objects.

The Spring Framework and Spring Boot use different software design conventions to structure the code and classes.


\subsection{RabbitMQ}
\label{sec:rabbitmq}
RabbitMQ\footnote{\url{https://www.rabbitmq.com/}} is an open-source message broker that supports different messaging protocols like MQTT, STOMP, and AMQP.
The system supports a variety of asynchronous messaging techniques \eg, delivery acknowledgment and flexible routing.

Since RabbitMQ is a widely-used message broker, the Spring framework (\ref{sec:spring}) already comes with the needed libraries to work with the RabbitMQ system.

In the context of the Basilisk platform, we only need the most basic functionalities of message queues with a single producer and a single consumer.
RabbitMQ ensures reliable delivery of messages by using acknowledgments and confirms of the consumers\footnote{\url{https://www.rabbitmq.com/reliability.html}}.
If a message is lost in transit or the consumer encounters an exception, the message is not deleted from the queue.
Only if the consumer sends a confirmation to the RabbitMQ broker the message is deleted from the queue.
The RabbitMQ broker also ensures the durability of the queues and messages through file backups.


\subsection{Docker}
Docker\footnote{\url{https://docs.docker.com/get-started/overview/}} is a platform for containerizing applications for development and deployment.
Following the idea of a shipping container used on freight, a Docker container can be installed on any system that supports the Docker technology.
That means that there are no other software requirements than the Docker engine to run a container.
Inside the container, all requirements and libraries are installed that are needed by the application that is run in the container.

The application can only reach the outside of the container and, vice-versa, can be reached from the outside, if the required ports are published.
This encapsulates the application from unwanted accesses, and protects the host system from malicious software that could be run inside a container.
\\

A Docker container is always built from a Docker image.
A Docker image is a read-only template to create a Docker container with the running application inside.

There are two main ways to create a Docker image.
First, it could be downloaded from \dockh{}.
The second option is to build the Docker image from a Dockerfile.
A Dockerfile\footnote{\url{https://docs.docker.com/engine/reference/builder/}} contains the instructions for building a Docker image.
The Dockerfile first references the base image, which is used to build the image.
This could be, for example, a standard Ubuntu distribution.
After the base image, the Dockerfile defines which commands are run or which files should be copied from the host system into the image.

After the image is built or downloaded from \dockh{}, the container can be further configured before it gets started.
For example, ports can be published to make the service running inside the container available to the outside.
\\

Often multiple containers are needed to deploy a complete service platform.
For example, the Basilisk platform consists of three containerized services, a RabbitMQ container, and a container running Fuseki.
For these situations, Docker offers Docker Compose\footnote{\url{https://docs.docker.com/compose/}} files, which help to orchestrate multiple containers in a Docker network.



%\subsection{Software Design Patterns}
%\label{sec:software_design_patterns}
%Software design patterns are used by developers to structure code.
%The idea is that code of different projects share reusable structures of best practices for commonly occurring implementation problems.
%Further more patterns are used to improve the understanding and readability of the code\cite{fowlerPatternsEnterpriseApplication2003}.
%
%As stated in section \ref{sec:spring} Spring Boot requires the usage of different software design paradigms and patterns which give the source code a common structure.
%
%The following sections introduce design patterns which are used in the source code of the Basilisk platform.
%
%
%\subsubsection{Repository Pattern}
%The repository pattern is used to abstract the interaction to a data storage.
%This data storage can be an relational database or a data storage of any kind\cite{fowlerRepositoryPatternHttps, bansalDAOVsRepository2020}.
%
%The business logic of an application only interacts with a so called repository without actually knowing which kind of data storage is used.
%The repository then handles the storage and access of objects with the data storage.
%
%This has the advantage that the actual storage system can easily be changed.
%When the storage system is changed, only the repository class needs to be adjusted.
%The business logic can still use the same interface of the repository class to access its data.
%
%\subsubsection{Domain Driven Design}
%Domain driven 
%
%
%\todo[inline]{fill}

%\subsection{Stateful / Stateless Microservices}
%\todo[inline]{fill}


