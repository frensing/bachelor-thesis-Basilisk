\section{Semantic Web}
\label{sec:sw_topics}
The following topics stem from the research area of the Semantic Web.
Semantic Web is the research field that tries to represent information on the internet in a way that makes it processable by computers \cite{hitzlerSemanticWebGrundlagen2008}.

Since this thesis focuses mostly on the implementation and deployment of the Basilisk platform, these topics are mainly introduced to give a basic understanding to the context in which the Basilisk platform will be used.

\subsection{Knowledge Graphs} 
\label{sec:knowledge_graphs}
Knowledge Graphs are graphs intended to represent knowledge of the real world or small scenarios.
The knowledge stored in Knowledge Graphs is modeled in a graph-based structure. 
Nodes represent entities that are connected by various types of relations, represented by labeled edges in the graph.
This has the benefit of being able to represent complex relations between different nodes and edges \cite{hoganKnowledgeGraphs2021}.

The most straightforward Knowledge Graph consists of three elements:
The subject entity, the object entity, and the labeled edge between them describing their relation.
The relation is also called the predicate.
This atomic data entity consisting of subject, predicate, and object is called \emph{triple}.
In figure \ref{fig:example-knowledge-graph}, a simple example of a Knowledge Graph is shown.

\begin{figure}[tbph]
	\centering
	\includegraphics[width=0.4\textwidth]{figures/knowledge-graph-diagram}
	\caption{Simple Knowledge Graph}
	\label{fig:example-knowledge-graph}
\end{figure}

This Knowledge Graph could be extended in all directions.
For example, the "Berlin" entity could get an edge labeled "population" which would point to Berlin's population size.

Since a graph structure is hard to store in a classic relational database, a different storage types are needed.
The particular kind of database developed to store knowledge graphs is called \tsp{}.
Triplestores will be explained in section \ref{sec:triplestores}.


\subsection{RDF}
\label{sec:rdf}
The \acf{rdf} is a framework for describing data and knowledge in a standardized way \cite{RDFConceptsAbstract}. 
It is part of the W3C standard.
The information is written down as subject-predicate-object triples, representing the basic structure that is also present in Knowledge Graphs (\ref{sec:knowledge_graphs}).
The elements of those triples can be \acp{iri}, blank nodes or data typed literals.

\ac{rdf} graphs can be encoded with different syntax styles.
A popular syntax is TURTLE \cite{RDFTurtle}, which is a compact way of writing down a \ac{rdf} graph structure.
Using the example of section \ref{sec:knowledge_graphs}, the knowledge graph would be represented with the TURTLE syntax as seen in listing \ref{lst:rdf_turtle}.
The first two lines of the TURTLE document define abbreviations for the used \acp{iri} so that the triple in line three is more readable.

\begin{lstlisting}[caption={RDF example in TURTLE syntax}, label=lst:rdf_turtle]
@prefix dbr: <http://dbpedia.org/resource/> .
@prefix dbo: <http://dbpedia.org/ontology/> .
dbr:Germany dbo:capital dbr:Berlin .
\end{lstlisting}


\subsection{Triplestore}
\label{sec:triplestores}
Triplestores are a special kind of database developed to easily store and access knowledge graphs through queries \cite{rusherTriplestoresHttpsWww}.
They differ from relational databases by being purposefully built to only store and access data in a set of triples.
Data is accessed through the query language SPARQL, which gets introduced in section \ref{sec:sparql}.
Often more extensive datasets can be imported or exported using \ac{rdf} or another syntax.

Example of \tsp{} are \tentris{}\footnote{\url{https://tentris.dice-research.org/}}, GraphDB\footnote{\url{https://graphdb.ontotext.com/}}, Virtuoso\footnote{\url{https://virtuoso.openlinksw.com/}}, or Jena TDB\footnote{\url{https://jena.apache.org/documentation/tdb/}}.
Since \tentris{} is actively developed by our research group, we will focus our tests and evaluations of the platform on this \ts{}.

This thesis focuses on \tsp{} that has a SPARQL endpoint and accept SPARQL queries since the used benchmark framework \iguana{} is using the SPARQL endpoint to perform benchmarks (see section \ref{sec:iguana}).


\subsection{SPARQL}
\label{sec:sparql}
SPARQL (SPARQL Protocol and RDF Query Language) \cite{harrisSPARQLQueryLanguage} is a query language for manipulating and retrieving RDF data stored in \tsp{}.
Like RDF, SPARQL is part of the W3C recommendations for technologies in the semantic web.

The syntax for SPARQL queries looks similar to the SQL syntax since its main parts are also a \texttt{SELECT} clause, stating which variables to query for, followed by a \texttt{WHERE} clause giving restrictions and conditions.

Queries can contain optional graph patterns, conjunctions, disjunctions, and aggregation functions.
These extensions can help formulate more complex queries.

There are two example SPARQL queries in listings \ref{lst:sparql1} and \ref{lst:sparql2}.
Executed against the DBPedia SPARQL endpoint\footnote{\url{https://dbpedia.org/sparql}}, the following results can be found:
The first example query (\ref{lst:sparql1}) requests the variable which matches the \texttt{WHERE} clause searching for the capital of Germany, which is \texttt{dbr:Berlin}.
The second query (\ref{lst:sparql2}) requests all relationships that can be found between Germany and Berlin, which will return \texttt{dbo:capital}, which we expected, but also \texttt{dbo:wikiPageLink}, which means that there is a link from the Wikipage of Germany to the Wikipage of Berlin.

\begin{lstlisting}[caption={SPARQL query searching the capital of Germany}, label=lst:sparql1]
PREFIX dbr: <http://dbpedia.org/resource/>
PREFIX dbo: <http://dbpedia.org/ontology/>

SELECT ?capital
WHERE {
	dbr:Germany dbo:capital ?capital .
}
\end{lstlisting}

\begin{lstlisting}[caption={SPARQL query searching all relations between Germany and Berlin}, label=lst:sparql2]
SELECT ?relation
WHERE {
	dbr:Germany ?relation dbr:Berlin .
}
\end{lstlisting}