\chapter{Implementation}
\label{ch:implementation}



This section describes the implementation of the concepts explained in chapter \ref{ch:approach}.
Parts of the system were already implemented by other developers before this thesis.

As stated in section \ref{sec:architecture_review} some parts of the functionality are still missing and need to be implemented.
This chapter documents the design and implementation of the explained shortcomings.
\\

In section \ref{sec:solution_design} we describe how we design the missing parts and how we plan to implement them into the platform.
In section \ref{sec:solution_implementation} we document the implementation process.
Possible problems or new insights found during this process are explained and dealt with.


\todo[inline]{Präsenz oder Vergangenheitsform?}


\todo[inline]{Projekt management? Agile Software entwicklung?}


\section{Thesis Time Schedule}
\label{sec:time_schedule}
The time schedule of the thesis had to be altered to allow more time for designing and implementing the Basilisk platform.

Before starting this thesis it was hard to quantify how much implementation work was still needed.
After the architecture review (\ref{sec:architecture_review}) it became clear that the implementation workload was greater than anticipated in the thesis proposal.

Therefore we an alteration of the time schedule of the thesis is needed.
Further we discussed the task priorities.
First priority is the development of the core functionality for the platform.
This means the main services are required to successfully perform a benchmark job and save the measured metrics to a \ts{}.

With a lower priority, functionalities like user management (\ref{sec:review_user_management}) will be dealt with.

The least priority has the Basilisk frontend (\ref{sec:basilisk_frontend}), since it is not needed to successfully run the platform.
Secondly it introduces more programming languages and frameworks to the project.
The time schedule can not provide enough time to fully understand this new technology stack.



\section{Solution Design}
\label{sec:solution_design}




\section{Solution Implementation}
\label{sec:solution_implementation}


In this section we give an high-level description of the implementation process.
During the implementation process the improvements and extensions that was found in section \ref{sec:architecture_review} and designed in section \ref{sec:solution_design} are implemented into the code base.
\\

The following sections follow the implementation process



\subsection{Code Refactoring}
\label{sec:impl_code_refactor}
As stated in section \ref{sec:code_refactor}, an in-depth code refactoring is recommended.
We start the implementation with this refactoring to create a clean code base for all future implementations.

Code refactoring is the process of restructuring the source code of an application without changing its functionality\cite{fowlerRefactoringImprovingDesign2019a}.




\section{Deployment}
\label{sec:deployment}
After the implementation phase, the Basilisk platform is deployed on a virtual server.
The virtual server used is provided by the IRB (Informatik Rechnerbetrieb) of the computer science department at Paderborn University.
Figure \ref{fig:vm_specs} show the specification of the VM.
The VM was choosen to be powerful and to have a lot of memory and storage.
Database benchmarks are often memory intensive and require a lot of storage for the different datasets.


\begin{figure}[tbph]
	\centering
	\begin{tabular}{ll}
		\toprule
		\textbf{Specifications} &                             \\ \midrule
		CPUs                    & 16 cores                    \\ \midrule
		Memory                  & 128GB                       \\ \midrule
		Storage                 & 4.17TB                      \\ \midrule
		Operating System        & Debian GNU/Linux 11 (64bit) \\ \bottomrule
	\end{tabular}
	\caption{Specification of the server on which the platform is deployed.}
	\label{fig:vm_specs}
\end{figure}

%%%%%%%%%%%%%%%%
% - Solution design and implementation
% - Deployment of the service
% - Implement benchmarking using IGUANA

% - Design patterns
	