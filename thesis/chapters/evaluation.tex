\chapter{Evaluation}
\label{ch:evaluation}

This chapter evaluates the added value to the benchmarking process that is created through the Basilisk platform.
\\

The main goal of the platform is to simplify the process of benchmarking a known \tsp{} and to automate the detection and the execution of a benchmark for new releases of configured \tsp{}.
Therefore we compare the setup process for a benchmark performed by the Basilisk platform to the process of executing a benchmark manually using the \iguana{} framework.

To Benchmark the capabilities of the platform, we setup continuous benchmarks for two different \tsp{}.
The first \ts{} is \tentris{}\cite{bigerlTentrisTensorBasedTriple2020} which is developed by the DICE-research group.
The second \ts{} benchmarked in this thesis is Oxigraph\footnote{\url{https://github.com/oxigraph/oxigraph}}.

\section{Initial Benchmark Setup}
In this section we compare the steps needed for setting up an initial \ts{} benchmark using the \iguana{} framework.
In general the execution of a benchmark requires a running \ts{}, the \iguana{} framework and a dataset and query file.

The next two sections describe the initial setup for a manual benchmark run and the recommended process to create a Basilisk configuration for a \ts{}.


\subsection{Manual Benchmark Setup}
\label{sec:eval_manual_benchmark_setup}
To manually run a benchmark, first the \ts{} needs to be installed and started.
For a manual testrun this can be done either as a full installation or by using a Docker container.
Often it is easier to use a ready-to-run docker container that contains all needed dependencies and a running installation of the \ts{}.
When the \ts{} is running, the dataset needs to be loaded into the \ts{}.
This can be done either by providing the dataset during startup or by uploading the data through the SPARQL endpoint.
Lastly the \iguana{} framework needs to be configured by providing a configuration file containing the query file and SPARQL endpoint.


\subsection{Basilisk Benchmark Setup}
When a \ts{} is fully configured in the Basilisk platform, the platform will automatically provide all four requirements for an benchmark, when a new job is created.

To create a working configuration for the platform we recommend to develop and test a local setup first.
The process to create this initial test setup is similar to the setup of a manual benchmark explained in section \ref{sec:eval_manual_benchmark_setup}.
But in this case we need to use a Docker container, since Basilisk is only working with a container setup.

The local setup should consist of a \ts{} running in an Docker container, which is also reachable over the SPARQL endpoint.
To make sure that \iguana{} is able to perform an benchmark, it is also advised to start a short benchmark with a simple \iguana{} configuration.
When a working setup is found, the setup can be transferred into the Basilisk platform.


\subsection{Comparison}
The initial setup to perform one benchmark for one \ts{} version is the same for the manual process aswell as for the Basilisk process.
In both scenarios the \ts{} and \iguana{} is setup and run manually.
Additionally, the configuration needs to be transferred to the Basilisk platform before a benchmark can actually be started.


\section{Setup of further Benchmarks}
The real advantage of the Basilisk platform can be seen when further benchmarks have to be run on an already configured \ts{}.
This can be a benchmark using a different dataset and query file or it can be a benchmark of a different version of a \ts{}.

Both scenarios are evaluated in the following sections.

\subsection{Using a different Benchmark}
In the scenario of using a different benchmark, a new dataset and query file are used.
For the manual setup described in section \ref{sec:eval_manual_benchmark_setup} multiple steps have to be done to update the dataset and query file.
First The dataset needs to be loaded into the \ts{} and the \iguana{} configuration needs to be adjusted to user the new query file.
\\

In case of the Basilisk setup, only the new dataset and query file have to be configured in the platform.
When a new benchmark job is executed, the \iguana{} configuration is automatically generated using the new benchmark setup.


\subsection{Benchmarking a new Version}
If a new version of a \ts{} should be benchmarked, again multiple steps are needed for the manual benchmark process.
First the new version has to be downloaded and started as a Docker container.
Then the dataset needs to be loaded into the \ts{} and lastly the \iguana{} configuration needs to be updated to the new SPARQL endpoint location.
\\

The Basilisk configuration does not need to be changed.
If a new version has to be benchmarked, either the platform has already noticed the new version and created a new benchmark job automatically or the user can create a manual benchmark job by providing the benchmark that should be used and the \ts{} version.
The platform will then automatically setup the container and configure \iguana{} to run the benchmark job.
Of course, this will only work if the other versions of the \ts{} can use the same basic configuration for startup, loading the dataset and providing the SPARQL endpoint.




%%%%%%%%%%%%%%%%%%%%%%%%%
% - Experiment setup, requirements
% - Performing of benchmarks
% - Result evaluation

%- was macht basilisk was vorher nicht da war
%- versch systeme mit selben zweck vergleichen
%	- suche? eigentlich nicht verfügbar
%- richtung: einfaches setting
%	- benchmark
%	- 3 triplestores
%	- wie lange mit system / ohne system
%- schnell im basilisk zw benchmarks wechseln
%- im voraus planen was zu testen

% - ansible tentris - https://github.com/dice-group/tentris-paper-benchmarks/releases/tag/v1.0
