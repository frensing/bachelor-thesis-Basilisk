\chapter{Evaluation}
\label{ch:evaluation}

This chapter evaluates the Basilisk platform based on the developments described in chapter \ref{ch:implementation} and the ease of using the platform to setup a continuous benchmark job for an existing repository.
We will compare the manual benchmarking process using \iguana{} to the use of the Basilisk platform and evaluate the added value the platform creates for the benchmarking process.
\\

The main goal of the platform is to simplify the process of benchmarking a known \tsp{}. 
The platform automates the detection of a new release of a configured \ts{} and  automates the execution of a benchmark job for the new releases.
%This was not accomplished by any other platform before.
%In our research we did not find a solution that offers the possibility of running automated benchmarks
\\

To evaluate the capabilities of the platform, we setup continuous benchmarks for two different \tsp{}.
The first \ts{} is \tentris{}\cite{bigerlTentrisTensorBasedTriple2020} which is developed by the DICE-research group.
The second \ts{} benchmarked in this thesis is Oxigraph\footnote{\url{https://github.com/oxigraph/oxigraph}}.

The \tsp{} are chosen because they both are available as a ready-to-run docker image on \dockh{}.
They also differ in the way the benchmark dataset is loaded into their internal storage.
The \tentris{} \ts{} accepts the dataset file as an argument at program start, while Oxigraph needs to be started before the data is upoaded through the SPARQL endpoint of the \ts{}.


\section{Initial Benchmark Setup}
In this section we compare the steps needed for setting up an initial \ts{} benchmark using the \iguana{} framework.
In general the execution of a benchmark has the following four requirements:
A running \ts{}, the \iguana{} framework, a dataset file and a query file.

The next two sections describe the initial setup for a manual benchmark run and the recommended process to create a Basilisk configuration for a \ts{}.


\subsection{Manual Benchmark Setup}
\label{sec:eval_manual_benchmark_setup}
To manually run a benchmark, first the \ts{} needs to be installed and started.
For a manual test run this can be done either as a full installation or by using a Docker container.
Often it is easier to use a ready-to-run docker container that contains all needed dependencies and a running installation of the \ts{}.
On the host system only the Docker engine is needed to run a container.

When the \ts{} is running, the dataset needs to be loaded into the \ts{}.
This can be done either by providing the dataset during startup or by uploading the data through the SPARQL endpoint.
Lastly the \iguana{} framework needs to be configured by providing a configuration file containing the query file and SPARQL endpoint.

This process is similar for \tentris{} and Oxigraph.
The only difference is in the upload of the dataset.


\subsection{Basilisk Benchmark Setup}
When a \ts{} is fully configured in the Basilisk platform, the platform will automatically provide all four requirements for an benchmark, when a new benchmark job is automatically created.

To create a working \ts{} configuration for the platform we recommend to develop and test a local setup first.
The process to create this initial test setup is similar to the setup of a manual benchmark explained in section \ref{sec:eval_manual_benchmark_setup}.
But in this case we need to use a Docker container, since Basilisk is only working with a container setup.

The local setup should consist of a \ts{} running in an Docker container, which is also reachable over the SPARQL endpoint.
To make sure that \iguana{} is able to perform an benchmark, it is also advised to start a short benchmark with a simple \iguana{} configuration.

The \iguana{} configurations for the tested \tsp{} will slightly differ for loading the dataset to the \ts{}.
In case of \tentris{}, the dataset is configured to be provided inside the Docker container to be loaded on startup.
For Oxigraph, the dataset does not need to be provided inside the Docker container.
For the situation that the dataset needs to be loaded after the startup, \iguana{} needs a pre-hook-script that will be executed before the real benchmark starts.
This script needs to be implemented and tested with the local test setup.

When a working setup is found, the setup can be transferred into the Basilisk platform.
Again, the setup for \tentris{} and Oxigraph are mostly the same.
For Oxigraph, the custom load script is provided and the Basilisk configuration will point to the script when creating an \iguana{} configuration.



\subsection{Comparison of Initial Setups}
The initial setup to perform one benchmark for one \ts{} version is nearly the same for the manual process aswell as for the Basilisk process.
In both scenarios the \ts{} and \iguana{} is setup and run manually.
The configuration of the Basilisk platform is more complicated for the case of loading the dataset through the SPARQL endpoint, since a custom load-script is needed.
Additionally, the configuration needs to be transferred to the Basilisk platform before a benchmark can actually be started.


\section{Setup of further Benchmarks}
The real advantage of the Basilisk platform can be seen when further benchmarks have to be run for an already configured \ts{}.

We look into two scenarios that require the run of further benchmarks on a known \ts{}.
Both scenarios will be looked at for the manual setup and a already configured Basilisk setup.
The first scenario is the usage of a different dataset and query file as a new benchmark that is to be run.
The second scenario is the benchmark of a different version of a configured \ts{}.

Both scenarios are evaluated in the following sections.


\subsection{Using a different Benchmark}
In the scenario of using a different benchmark, a new dataset and query file are used.
For the manual setup described in section \ref{sec:eval_manual_benchmark_setup} multiple steps have to be done to update the dataset and query file.
First The dataset needs to be loaded into the \ts{}.
This can be done by using the SPARQL endpoint to upload the data or by restarting the \ts{} and providing the new dataset at startup.
For \tentris{} it is usually easier to restart the \ts{} with the new dataset as a argument on startup.
Secondly the the \iguana{} configuration needs to be adjusted to use the new query file.
\\

In case of the Basilisk setup, only the new dataset and query file have to be configured in the platform.
When a new benchmark job is executed, the \iguana{} configuration is automatically generated using the new benchmark setup.
If the load-script for Oxigraph is setup correctly the new dataset will also be automatically uploaded to the \ts{}.
To perfrom the new benchmark, a manual job can be started by sending a request to the API of the \ac{jms}


\subsection{Benchmarking a new Version}
If a new version of a \ts{} should be benchmarked, again multiple steps are needed for the manual benchmark process.
First the new version has to be downloaded and started as a Docker container.
Then the dataset needs to be loaded into the \ts{} and lastly the \iguana{} configuration needs to be updated to the new SPARQL endpoint location.
\\

The Basilisk configuration does not need to be changed.
If a new version has to be benchmarked, either the platform has already noticed the new version on its own and created a new benchmark job automatically, or the user can create a manual benchmark job by providing the benchmark that should be used and the \ts{} version to the API of the \ac{jms}.
The platform will then automatically setup the container and configure \iguana{} to run the benchmark job.
This is the main idea why the platform was originally developed.
Of course, this will only work if the other versions of the \ts{} can use the same basic configuration for startup, loading the dataset and providing the SPARQL endpoint.
If there are major changes to the setup and structure of a \ts{}, a new configuration in Basilisk is needed.


\subsection{Comparison of Benchmark Changes}
As seen in the description of the above scenarios, the \ts{} configuration of the Basilisk platform is not changed at all.
Only the new benchmark files are registered in case a new benchmark should be performed.

In contrast to this, the manual setup requires a lot of manual changes.
Each change to a running configuration, basically requires the user to setup the benchmark configuration from the start.
Either the \ts{} is downloaded and setup again, or the \iguana{} configuration needs manual changes.


\section{Basilisk Evaluation}
The comparison of the Basilisk platform to the manual benchmark process already shows the advantage of using Basilisk to perform multiple benchmarks on different \ts{} versions.

The initial setup of a \ts{} for the Basilisk platform is more complicated and a little more restricted than the manual setup.
But as soon as the second benchmark is performed, the Basilisk platform has no manual setup time by the user.
Once the \ts{} is configured, either the benchmark jobs are created automatically by the system, or the user can manually create a job.

Using the system we were able to perform 16 benchmarks on different \tentris{} versions using the SWDF dataset in just one hour.


\todo[inline]{graphics of benchmarks.}


%This is also the first time the Oxigraph \ts{} is benchmarked independently ??

The idea of the Basilisk platform is focused on the small use-case of automatically benchmarking configured \tsp{} through their SPARQL endpoints.
Basilisk fulfills its task of automating the benchmark process of known \ts{}.




%%%%%%%%%%%%%%%%%%%%%%%%%
% - Experiment setup, requirements
% - Performing of benchmarks
% - Result evaluation

%- was macht basilisk was vorher nicht da war
%- versch systeme mit selben zweck vergleichen
%	- suche? eigentlich nicht verfügbar
%- richtung: einfaches setting
%	- benchmark
%	- 3 triplestores
%	- wie lange mit system / ohne system
%- schnell im basilisk zw benchmarks wechseln
%- im voraus planen was zu testen

% - ansible tentris - https://github.com/dice-group/tentris-paper-benchmarks/releases/tag/v1.0
