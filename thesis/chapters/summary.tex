\chapter{Summary and Discussion}
\label{ch:summary}

In this chapter we summarize our development of the Basilisk platform and highlight the key findings of the evaluation phase.
Lastly we point out the future work that can be done to extend the functionality of the platform.

\section{Summary}
In this thesis we have continued the development of the Basilisk platform.
The main task of the platform is to continuously check for new \ts{} releases and automatically perform benchmarks if a new release is found.
This functionality was successfully implemented for \dockh{} repositories and tested and evaluated with the \tentris{} and Oxigraph \tsp{}.



\section{Future Work}
The development performed in this thesis has resulted in a running version of the Basilisk platform.
Currently the benchmark process for \dockh{} repositories is working and benchmarks are successfully performed.

Because the thesis time schedule had to be altered during the implementation process, some functionalities could not been fully implemented.
Most of these missing functionalities are not strictly relevant to successfully run the platform, but would be a nice-to-have from a user perspective.

These functionalities are for example, a user management system, extended \ts{} configurations, and the Basilisk frontend.
The user management system could manage different user roles and access rights for the various system functionalities.
For example some users could be only allowed to view the job status or create a manual job, while admin user could change the \ts{} configurations.
The extended \ts{} configurations could define a range or list of version for which a configuration is valid for use.
If for example a newer \ts{} version requires the dataset to be loaded differently, a new configuration could be setup to be used with that version.
Lastly the Basilisk frontend would provide a fast and easy way to interact with the benchmark results and could also offer a user interface for setting up and managing the configured \ts{} and benchmarks.

The most important functionality that is still missing in the platform is the performing of benchmark jobs on \gh{} repositories.
As stated in section \ref{sec:time_schedule} the functionalities for observing \gh{} repositories and creating benchmark jobs are already implemented.
What is still missing is the functionality for downloading the source code from \gh{} and building a Docker image from the source code files.
After that the process of starting the image and running the benchmark will be the same as with \dockh{} repositories.





%%%%%%%%%%%%%%%%%%%%%%%%%%
% - Summary of the work
% - Highlighting the key findings of the evaluation stage