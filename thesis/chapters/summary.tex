\chapter{Summary and Discussion}
\label{ch:summary}

In this chapter we summarize our development of the Basilisk platform and highlight the key findings of the evaluation phase.
Lastly we point out the future work that can be done to extend the functionality of the platform.

\section{Summary}
Benchmarking of \tsp{} is in general very time consuming and often requires a lot of configuration and setup time.
Because of this, \tsp{} are often benchmarked late in the development process.
To improve this time consuming process and to try to automate the benchmark process as much as possible, the development of the Basilisk platform was started.
The main task of the platform is defined to continuously benchmark \tsp{}.
This means that the platform is configured once and will then automatically perform benchmarks on new versions of the configured \tsp{}.
\\

In this thesis we have continued the development of the Basilisk platform.
The platform is designed to continuously check for new \ts{} releases and to automatically perform benchmarks if a new release is found.
\ts{} releases could be found either in \dockh{} or \gh{} repositories.
The complete process of checking for new releases and benchmarking new releases is fully implemented for \dockh{} repositories.
\\

During this thesis we have performed an in-depth architecture and code review of the platform.
Based on that review we defined implementation tasks for improving and extending the existing code base.

The main improvements were the restructure of the microservices and a restructure of the used data models.
These changes were implemented before starting on the missing implementations of the benchmark service.

Most of the functionality of the \acl{tbs} was not yet implemented.
The implemented functionality initializes the \ts{} inside a Docker container and performs the benchmarks.
During this thesis we focused on completing the benchmark process for \ts{} releases in \dockh{} repositories.
To provide the same functionality for \gh{} repositories some further steps are needed which are explained in section \ref{sec:future_work}.

To perform the benchmarks, the benchmark-independent \iguana{} framework is used.
The configuration for \iguana{} is generated from the information stored in the \ts{} configurations inside the Basilisk platform.

Finally the Basilisk platform was deployed on a VM hosted inside the network of the Paderborn University.
On this deployment the \tentris{} and Oxigraph \tsp{} were configured and multiple versions of both \tsp{} were successfully benchmarked.
The user only needs to start the system and to configure the \tsp{}.
No further interactions are needed during the benchmark process.
When the benchmarks are completed, the results are available in the \acl{jsts}.



\section{Future Work}
\label{sec:future_work}
The development performed in this thesis has resulted in a running version of the Basilisk platform.
Currently the benchmark process for \dockh{} repositories is working and benchmarks are successfully performed.

Because the thesis time schedule had to be altered during the implementation process, some functionalities could not been fully implemented.
Most of these missing functionalities are not strictly relevant to successfully run the platform, but would be a nice-to-have from a user perspective.

These functionalities are for example, a user management system, extended \ts{} configurations, and the Basilisk frontend.
The user management system could manage different user roles and access rights for the various system functionalities.
For example some users could be only allowed to view the job status or create a manual job, while admin user could change the \ts{} configurations.
The extended \ts{} configurations could define a range or list of version for which a configuration is valid for use.
If for example a newer \ts{} version requires the dataset to be loaded differently, a new configuration could be setup to be used with that version.
Lastly the Basilisk frontend would provide a fast and easy way to interact with the benchmark results and could also offer a user interface for setting up and managing the configured \ts{} and benchmarks.

The most important functionality that is still missing in the platform is the performing of benchmark jobs on \gh{} repositories.
As stated in section \ref{sec:time_schedule} the functionalities for observing \gh{} repositories and creating benchmark jobs are already implemented.
What is still missing is the functionality for downloading the source code from \gh{} and building a Docker image from the source code files.
After that the process of starting the image and running the benchmark will be the same as with \dockh{} repositories.





%%%%%%%%%%%%%%%%%%%%%%%%%%
% - Summary of the work
% - Highlighting the key findings of the evaluation stage