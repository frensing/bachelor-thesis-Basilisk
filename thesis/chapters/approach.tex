\chapter{Approach}
\label{ch:approach}
%%%%%%%%%%%%%%%%%%%%%%%
% - description of the designed system
% - analysis and review of the current software architecture
% - gerne in die Tiefe gehen
%%%%%%%%%%%%%%%%%%%%%%%

In this chapter we give an overview of the current software architecture on which the Basilisk platform is build.

The basic architecture pattern of the Basilisk platform is the microservice architecture (see chapter \ref{sec:microservice_architecture} for a short description). 
This means that the platform is divided into multiple services on which the workload and the different tasks are divided.
The services can be run on different hardware systems and they interact with each other via a message queue system.
\\


Figure \ref{fig:basilisk_high_level_design} gives an overview of the microservice architecture for the Basilisk platform and the most important messages send between the services.
\begin{figure}[tbph]
	\centering
	\includegraphics[width=1.1\textwidth]{figures/basilisk_high_level_design.pdf}
	\caption{High level design of the Basilisk framework}
	\label{fig:basilisk_high_level_design}
\end{figure}
\\


\section{Main Services}
\label{sec:main_services}
The next sections explain the three main services, namely Hooks Checking Service (\ref{sec:hooks_checking_service}), Jobs Managing Service (\ref{sec:jobs_managing_service}), and \ts{} Benchmarking Service (\ref{sec:ts_benchmarking_service}).


\subsection{Hooks Checking Service}
\label{sec:hooks_checking_service}
The main task of the hooks checking service is to observe Github and Dockerhub repositories for new releases or changes.

\subsubsection{API}
\label{sec:hooks_api}


\subsection{Jobs Managing Service}
\label{sec:jobs_managing_service}
The Jobs Managing Service processes the requests coming from the web-frontend, checks if the Hooks Checking Service has found a new version for a benchmark and creates jobs for new benchmarks.

\subsubsection{API}
\label{sec:jobs_api}


\subsection{\ts{} Benchmarking Service}
\label{sec:ts_benchmarking_service}
Lastly the \ts{} Benchmarking Service executes the benchmarks given to it and saves the results to a database.

\subsubsection{API}
\label{sec:benchmarking_api}


\section{Message Queue System}
\label{sec:message_queue}
The messages send between the services are transmitted via the RabbitMQ\footnote{\url{https://www.rabbitmq.com/}} message broker software.


\section{Programming Language and Frameworks}
\label{sec:prog_lang_and_framework}
All services are implemented with Java and are using the Spring Boot framework.

\subsection{Package Structure}
\label{sec:folder_structure}
The package structure used for the service implementation is similar in all three services.

\paragraph{config}
In the config folder important Spring Beans are defined and configured.
Also the configuration for the RabbitMQ message queues is defined here.
The classes in this package are annotated with the @Configuration marker which informs Spring that these classes declare @Bean methods for generating Spring Beans.
Beans are defined in Spring as the objects that form the backbone of the application.

\paragraph{core}



\section{Frontend}
\label{sec:frontend}