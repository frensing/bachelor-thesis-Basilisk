The process of benchmarking \tsp{} can be very time-consuming.
When a new version of a \ts{} is released, it has to be installed and setup in order to perform a benchmark on it.

The idea of the Basilisk platform is, that the process of benchmarking new \ts{} releases gets fully automated.
To accomplish this, the platform can observe \dockh{} and \gh{} repositories.
When a new release is detected, a benchmark job is created.
The platform automatically downloads the new release, sets up a Docker container and configures the \iguana{} framework before starting the benchmark.
After the benchmark, the results measured by \iguana{} are stored in the result \ts{}.

During this thesis, we analyzed and reviewed the existing implementations of the Basilisk platform.
We then implemented the entire benchmark process for repositories on \dockh{} and prepared the benchmark process for \gh{} repositories.
Finally we evaluated the automated benchmark process and compared it to a manual setup using the \iguana{} framework.

